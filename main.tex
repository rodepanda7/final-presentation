\documentclass{beamer}
\usetheme{metropolis}           % Use metropolis theme
\title{Formalizing Modal Embeddings of Call-by-Name and Call-by-Value}
\date{\today}
\author[Floris Reuvers]{Floris Reuvers \\[1ex]
  \small Supervisor: dr. N.M. van der Weide \\
  \small Second Reader: dr. E.G.M. Hubbers
}
\institute{Radboud University} % What is actually the institute?

\usepackage[utf8]{inputenc}
\usepackage[T1]{fontenc}
\usepackage{graphicx}
\usepackage{array}
\usepackage{listings}
\usepackage{rutitlepage}
\usepackage{hyperref}
\usepackage{enumitem}
\usepackage{amsmath}
\usepackage{amssymb}
\usepackage{xspace}
\usepackage{tabularx}
\usepackage{multicol}
\usepackage{multirow}
\usepackage{float}
\usepackage{cmll}
\usepackage{lipsum}
\usepackage{amsthm}
\usepackage{thmtools}
\usepackage{booktabs}
\usepackage{caption}
\usepackage{cleveref}
\usepackage{amsthm}
\usepackage{enumitem}
\usepackage{latexsym}
\usepackage{newunicodechar}
\usepackage{code/latex/agda} % path to generated agda.sty

\newcommand{\lambdabar}{\lambda\!\!\raisebox{0.6ex}{\scalebox{0.8}{--}}}

\newunicodechar{□}{\ensuremath{\mathnormal\square}}
\newunicodechar{⇒}{\ensuremath{\mathnormal\Rightarrow}}
\newunicodechar{λ}{\ensuremath{\mathnormal\lambda}}
\newunicodechar{↝}{\ensuremath{\mathnormal\leadsto}}
\newunicodechar{₁}{\ensuremath{_1}}
\newunicodechar{₂}{\ensuremath{_2}}
\newunicodechar{ₙ}{\ensuremath{_n}}
\newunicodechar{ν}{\ensuremath{\mathnormal\nu}}
\newunicodechar{ζ}{\ensuremath{\mathnormal\zeta}}
\newunicodechar{ξ}{\ensuremath{\mathnormal\xi}}
\newunicodechar{μ}{\ensuremath{\mathnormal\mu}}
\newunicodechar{β}{\ensuremath{\mathnormal\beta}}
\newunicodechar{⊢}{\ensuremath{\mathnormal\vdash}}
\newunicodechar{Γ}{\ensuremath{\mathnormal\Gamma}}
\newunicodechar{Δ}{\ensuremath{\mathnormal\Delta}}
\newunicodechar{∀}{\ensuremath{\mathnormal\forall}}
\newunicodechar{ƛ}{\ensuremath{\lambda\mkern-8mu\raisebox{0.45ex}{\rotatebox{15}{\scalebox{0.8}{--}}}}}


\RequirePackage{defs}
\RequirePackage{bussproofs}
\RequirePackage{formalgrammars}

\theoremstyle{definition}
% \newtheorem{theorem}{Theorem}[chapter] %QUESTION, SECTION OR CHAPTER?
% \newtheorem{definition}[theorem]{Definition}
% \newtheorem{lemma}[theorem]{Lemma}
% \newtheorem{proposition}[theorem]{Proposition}
% \newtheorem{example}[theorem]{Example}

% \crefname{definition}{Definition}{Definitions}
% \Crefname{definition}{Definition}{Definitions}

\begin{document}
  \maketitle
  %\section{Research Question}
  \begin{frame}{Research Question}
    \Large
    How can the unification of call-by-name and call-by-value evaluation strategies using modal logic be formalised in Agda?
  \end{frame}

  \section{Introduction}
  \begin{frame}{Context}
    \begin{itemize}
      \item[\textbullet] Functional Programming Languages
      \item[\textbullet] Evaluation Strategies
        \begin{itemize}
          \item[--] Call-by-value (cbv): OCaml/F\#/SML
          \item[--] Call-by-name (cbn)/Lazy Evaluation: Haskell/Clean
        \end{itemize}
      \item[\textbullet] Idea of cbn and cbv
        \begin{itemize}
          \item[--] cbv: first evaluate the argument, then the beta
          \item[--] cbn: evaluate the beta 
        \end{itemize}
    \end{itemize}
  \end{frame}
  \begin{frame}{Call-by-name}
    \begin{block}{Let f be defined as}
      \begin{flalign*}
        & f (x) = x * x &
      \end{flalign*}
    \end{block} \pause

    \begin{block}{cbn and cbv evaluation of f (3 + 3)} \pause
      \begin{align*}
        \onslide<3->{& \textsf{\textbf{cbn:}} &                    & \textsf{\textbf{cvb:}} &           } \\
        \onslide<4->{f (3 + 3)    & \to (3 + 3) * (3 + 3) & f (3 + 3)     & \to f (6) } \\
        \onslide<5->{             & \to 6 * (3 + 3)       &               & \to 6 * 6 } \\
        \onslide<6->{             & \to 6 * 6             &               & \to 36 }    \\
        \onslide<7->{             & \to 36}
      \end{align*}
    \end{block} \pause
  \end{frame}

  \begin{frame}{Reasons to unify cbn and cbv}
    \begin{itemize}
      \item[\textbullet] Reason 1
      \item[\textbullet] Reason 2
      \item[\textbullet] Reason 3
    \end{itemize}
    % \begin{itemize}
    %   \item[\textbullet] Why unify cbn and cbv?
    %   \item[\textbullet] Some approaches to unification:
    %     \begin{itemize}
    %       \item[--] Modal logic
    %       \item[--] Linear logic
    %       \item[--] Thunks
    %     \end{itemize}
    % \end{itemize}
  \end{frame}
  \begin{frame}{Approaches to unification}
    \begin{itemize}
      \item[\textbullet] Modal logic
      \item[\textbullet] Linear logic
      \item[\textbullet] Thunks
    \end{itemize}
  \end{frame}

  \section{Background}
  \begin{frame}{Grammar Lambda Calculus (\lc)}
    \begin{align*}
      \begin{grammar}{
        \pr{$A$}{$X \gors A \fa A'$}
      } \quad
      \end{grammar}
      \begin{grammar}{
        \pr{$M, N, P, Q$}{$x \gors \lambda x.M \gors M N$}
      }
      \end{grammar}
    \end{align*}
    \begin{block}{Example terms} \pause
      \begin{itemize}
        \item $x$ \pause
        \item $\lamb{x}{x}$ \pause
        \item $\lamb{y}{y} z$ \pause
        \item $(\lamb{y}{y}) z$ \pause
        \item $\lamb{x}{\lamb{t}{x t (\lamb{s}{w})}wy}$
      \end{itemize}
    \end{block}
  \end{frame}
  \begin{frame}{Call-by-name and call-by-value \lc}
    Lorem ipsum
  \end{frame}
  \begin{frame}{Closure Rules}
    Here we define closure rules
  \end{frame}
  \begin{frame}{Evaluation Relations}
    Lorem ipsum
  \end{frame}
  \begin{frame}{Call-by-box \lc}
    \begin{block}{Grammer}
      \begin{align*}
        \begin{grammar}{
          \pr{$A$}{$X \gors B \fa A \gors B$}
        } \quad \quad
        \end{grammar}
        \begin{grammar}{
          \pr{$B$}{$\boxit{A}$}
        }
        \end{grammar}
      \end{align*}
      \begin{align*}    
        \begin{grammar}{
          \pr{$M, N, P, Q$}{$\boxe{x} \gors \lambda x.M \gors M N \gors \boxi{N}$}
        }
        \end{grammar}
      \end{align*}
    \end{block}
  \end{frame}
  \begin{frame}{Evaluation Relations of \lab}
    Lorem ipsum
  \end{frame}

  \section{\texorpdfstring{Embeddings into \lab}{Embeddings into the boxed lambda calculus}}
  \begin{frame}{Girard's Translation}
    Lorem ipsum
  \end{frame}
  \begin{frame}{Gödel's Translation}
    Lorem ipsum
  \end{frame}

  \section{Challenges of Formalisation}
  \begin{frame}{Overview Challenges}
    \begin{itemize}
      \item[\textbullet] Variables
      \item[\textbullet] Ill typed terms
      \item[\textbullet] Formal definition of \raiseembn
    \end{itemize}
  \end{frame}
  \begin{frame}{De Bruijn Indices}
    Lorem ipsum
  \end{frame}
  \begin{frame}{Restriction to well-typed terms}
    Lorem ipsum
  \end{frame}
  \begin{frame}{Formal definition of \raiseembn}
    Lorem ipsum
  \end{frame}

  \section{Propositions}
  \begin{frame}{Girard's Translation}
    Lorem ipsum
  \end{frame}
  \begin{frame}{Gödel's Translation}
    Lorem ipsum
  \end{frame}

  \section{Conclusion}
  \begin{frame}{Conclusion}
    Lorem ipsum
  \end{frame}
  % \begin{frame}{Boxed Lambda Calculus}
  %   \small
  %   \begin{block}{Cbn \lc (\lan)}
  %     \begin{grammar}{
  %       \pr{$M, N, P, Q$}{$x \gors \lambda x.M \gors M N$}
  %     }
  %     \end{grammar}
  %     \begin{align*}
  %       (\lamb{x}{M})N \red M \subst{x}{N} \quad \quad (\bn)
  %     \end{align*}
  %   \end{block}
  %   \small
  %   \begin{block}{Boxed \lc (\lab)}
  %     \begin{grammar}{
  %       \pr{$M, N, P, Q$}{$\boxe{x} \gors \lambda x.M \gors M N \gors \boxi{N}$}
  %     }
  %     \end{grammar}
  %     \begin{align*}
  %       (\lamb{x}{M})\boxi{N} \red M \subst{\boxe{x}}{N} \quad \quad (\bb)
  %     \end{align*}
  %   \end{block}
  %   \small
  %   \begin{block}{Girard's Translation}
  %     \begin{align*}
  %       \girard{X}                 & = X                                          & \girard{x}             & = \boxe{x} \\
  %       \girard{(A_1 \fa A_2)}     & = \boxt{\girard{A_1}} \fa \girard{A_2}       & \girard{(\lamb{x}{M})} & = \lamb{x}{\girard{M}} \\
  %                                   &                                              & \girard{(M N)}         & = \girard{M} \boxi{\girard{N}}
  %     \end{align*}
  %   \end{block}
  % \end{frame}
  % \section{Research Results}
  % \begin{frame}[fragile]{Research Results}
  %   \begin{block}{Preservation of Reduction}
  %     \center  
  %     If $M \rednb N$ then $\girard{M} \redbbox \girard{N}$.
  %   \end{block}
  %   \tiny
  %   \begin{block}{Agda Proof}
  %     \begin{code}[hide]%
\>[0]\AgdaKeyword{module}\AgdaSpace{}%
\AgdaModule{nameProofs}\AgdaSpace{}%
\AgdaKeyword{where}\<%
\\
%
\\[\AgdaEmptyExtraSkip]%
\>[0]\AgdaKeyword{open}\AgdaSpace{}%
\AgdaKeyword{import}\AgdaSpace{}%
\AgdaModule{boxCalc}\<%
\\
\>[0]\AgdaKeyword{open}\AgdaSpace{}%
\AgdaKeyword{import}\AgdaSpace{}%
\AgdaModule{calc}\<%
\\
\>[0]\AgdaKeyword{open}\AgdaSpace{}%
\AgdaKeyword{import}\AgdaSpace{}%
\AgdaModule{embedCBNIntoCBB}\<%
\\
\>[0]\AgdaKeyword{open}\AgdaSpace{}%
\AgdaKeyword{import}\AgdaSpace{}%
\AgdaModule{nameCalcRed}\<%
\\
\>[0]\AgdaKeyword{open}\AgdaSpace{}%
\AgdaKeyword{import}\AgdaSpace{}%
\AgdaModule{boxCalcRed}\<%
\\
%
\\[\AgdaEmptyExtraSkip]%
\>[0]\AgdaKeyword{open}\AgdaSpace{}%
\AgdaKeyword{import}\AgdaSpace{}%
\AgdaModule{Data.Empty}\AgdaSpace{}%
\AgdaKeyword{using}\AgdaSpace{}%
\AgdaSymbol{(}\AgdaFunction{⊥}\AgdaSymbol{;}\AgdaSpace{}%
\AgdaFunction{⊥-elim}\AgdaSymbol{)}\<%
\\
\>[0]\AgdaKeyword{import}\AgdaSpace{}%
\AgdaModule{Relation.Binary.PropositionalEquality}\AgdaSpace{}%
\AgdaSymbol{as}\AgdaSpace{}%
\AgdaModule{Eq}\<%
\\
\>[0]\AgdaKeyword{open}\AgdaSpace{}%
\AgdaModule{Eq}\AgdaSpace{}%
\AgdaKeyword{using}\AgdaSpace{}%
\AgdaSymbol{(}\AgdaOperator{\AgdaDatatype{\AgdaUnderscore{}≡\AgdaUnderscore{}}}\AgdaSymbol{;}\AgdaSpace{}%
\AgdaInductiveConstructor{refl}\AgdaSymbol{;}\AgdaSpace{}%
\AgdaFunction{cong}\AgdaSymbol{;}\AgdaSpace{}%
\AgdaFunction{cong₂}\AgdaSymbol{;}\AgdaSpace{}%
\AgdaFunction{sym}\AgdaSymbol{)}\<%
\\
%
\\[\AgdaEmptyExtraSkip]%
\>[0]\AgdaKeyword{open}\AgdaSpace{}%
\AgdaModule{Eq.≡-Reasoning}\AgdaSpace{}%
\AgdaKeyword{using}\AgdaSpace{}%
\AgdaSymbol{(}\AgdaOperator{\AgdaFunction{begin\AgdaUnderscore{}}}\AgdaSymbol{;}\AgdaSpace{}%
\AgdaFunction{step-≡-∣}\AgdaSymbol{;}\AgdaSpace{}%
\AgdaFunction{step-≡-⟩}\AgdaSymbol{;}\AgdaSpace{}%
\AgdaOperator{\AgdaFunction{\AgdaUnderscore{}∎}}\AgdaSymbol{)}\<%
\\
\>[0]\AgdaKeyword{open}\AgdaSpace{}%
\AgdaModule{Eq.≡-Reasoning}\AgdaSpace{}%
\AgdaKeyword{renaming}\AgdaSpace{}%
\AgdaSymbol{(}\AgdaOperator{\AgdaFunction{begin\AgdaUnderscore{}}}\AgdaSpace{}%
\AgdaSymbol{to}\AgdaSpace{}%
\AgdaOperator{\AgdaFunction{beginEq\AgdaUnderscore{}}}\AgdaSymbol{;}\AgdaSpace{}%
\AgdaOperator{\AgdaFunction{\AgdaUnderscore{}∎}}\AgdaSpace{}%
\AgdaSymbol{to}\AgdaSpace{}%
\AgdaOperator{\AgdaFunction{\AgdaUnderscore{}∎eq}}\AgdaSymbol{)}\<%
\\
\>[0]\AgdaKeyword{open}\AgdaSpace{}%
\AgdaKeyword{import}\AgdaSpace{}%
\AgdaModule{Function.Bundles}\AgdaSpace{}%
\AgdaKeyword{using}\AgdaSpace{}%
\AgdaSymbol{(}\AgdaOperator{\AgdaFunction{\AgdaUnderscore{}⇔\AgdaUnderscore{}}}\AgdaSymbol{;}\AgdaSpace{}%
\AgdaFunction{mk⇔}\AgdaSymbol{)}\<%
\\
%
\\[\AgdaEmptyExtraSkip]%
\>[0]\AgdaComment{--\ GIRARD'S\ embedding}\<%
\\
%
\\[\AgdaEmptyExtraSkip]%
\>[0]\AgdaComment{--\ Preservation\ of\ typing\ by\ Girard's\ translation}\<%
\\
\>[0]\AgdaFunction{pres-typing}\AgdaSpace{}%
\AgdaSymbol{:}\AgdaSpace{}%
\AgdaSymbol{∀}\AgdaSpace{}%
\AgdaSymbol{\{}\AgdaBound{Γ}\AgdaSpace{}%
\AgdaBound{A}\AgdaSymbol{\}}\AgdaSpace{}%
\AgdaSymbol{→}\AgdaSpace{}%
\AgdaBound{Γ}\AgdaSpace{}%
\AgdaOperator{\AgdaDatatype{⊢}}\AgdaSpace{}%
\AgdaBound{A}\AgdaSpace{}%
\AgdaSymbol{→}\AgdaSpace{}%
\AgdaFunction{embedContext}\AgdaSpace{}%
\AgdaBound{Γ}\AgdaSpace{}%
\AgdaOperator{\AgdaDatatype{⊢b}}\AgdaSpace{}%
\AgdaFunction{embedType}\AgdaSpace{}%
\AgdaBound{A}\<%
\\
\>[0]\AgdaFunction{pres-typing}\AgdaSpace{}%
\AgdaBound{x}\AgdaSpace{}%
\AgdaSymbol{=}\AgdaSpace{}%
\AgdaFunction{embedTerm}\AgdaSpace{}%
\AgdaBound{x}\<%
\\
%
\\[\AgdaEmptyExtraSkip]%
\>[0]\AgdaFunction{ext-embed-lookup}\AgdaSpace{}%
\AgdaSymbol{:}\AgdaSpace{}%
\AgdaSymbol{∀}\AgdaSpace{}%
\AgdaSymbol{\{}\AgdaBound{Γ}\AgdaSpace{}%
\AgdaBound{Δ}\AgdaSpace{}%
\AgdaBound{A'}\AgdaSpace{}%
\AgdaBound{A}\AgdaSymbol{\}}\<%
\\
\>[0][@{}l@{\AgdaIndent{0}}]%
\>[2]\AgdaSymbol{→}\AgdaSpace{}%
\AgdaSymbol{(}\AgdaBound{f'}\AgdaSpace{}%
\AgdaSymbol{:}\AgdaSpace{}%
\AgdaSymbol{∀}\AgdaSpace{}%
\AgdaSymbol{\{}\AgdaBound{A}\AgdaSymbol{\}}\AgdaSpace{}%
\AgdaSymbol{→}\AgdaSpace{}%
\AgdaBound{A}\AgdaSpace{}%
\AgdaOperator{\AgdaDatatype{∈}}\AgdaSpace{}%
\AgdaBound{Γ}\AgdaSpace{}%
\AgdaSymbol{→}\AgdaSpace{}%
\AgdaBound{A}\AgdaSpace{}%
\AgdaOperator{\AgdaDatatype{∈}}\AgdaSpace{}%
\AgdaBound{Δ}\AgdaSymbol{)}\<%
\\
%
\>[2]\AgdaSymbol{→}\AgdaSpace{}%
\AgdaSymbol{(}\AgdaBound{fb'}\AgdaSpace{}%
\AgdaSymbol{:}\AgdaSpace{}%
\AgdaSymbol{∀}\AgdaSpace{}%
\AgdaSymbol{\{}\AgdaBound{Ab}\AgdaSymbol{\}}\AgdaSpace{}%
\AgdaSymbol{→}\AgdaSpace{}%
\AgdaBound{Ab}\AgdaSpace{}%
\AgdaOperator{\AgdaDatatype{∈b}}\AgdaSpace{}%
\AgdaFunction{embedContext}\AgdaSpace{}%
\AgdaBound{Γ}\AgdaSpace{}%
\AgdaSymbol{→}\AgdaSpace{}%
\AgdaBound{Ab}\AgdaSpace{}%
\AgdaOperator{\AgdaDatatype{∈b}}\AgdaSpace{}%
\AgdaFunction{embedContext}\AgdaSpace{}%
\AgdaBound{Δ}\AgdaSymbol{)}\<%
\\
%
\>[2]\AgdaSymbol{→}\AgdaSpace{}%
\AgdaSymbol{(\{}\AgdaBound{A'}\AgdaSpace{}%
\AgdaSymbol{:}\AgdaSpace{}%
\AgdaDatatype{Type}\AgdaSymbol{\}}\AgdaSpace{}%
\AgdaSymbol{(}\AgdaBound{x}\AgdaSpace{}%
\AgdaSymbol{:}\AgdaSpace{}%
\AgdaBound{A'}\AgdaSpace{}%
\AgdaOperator{\AgdaDatatype{∈}}\AgdaSpace{}%
\AgdaBound{Γ}\AgdaSymbol{)}\AgdaSpace{}%
\AgdaSymbol{→}\AgdaSpace{}%
\AgdaBound{fb'}\AgdaSpace{}%
\AgdaSymbol{(}\AgdaFunction{embedLookup}\AgdaSpace{}%
\AgdaBound{x}\AgdaSymbol{)}\AgdaSpace{}%
\AgdaOperator{\AgdaDatatype{≡}}\AgdaSpace{}%
\AgdaFunction{embedLookup}\AgdaSpace{}%
\AgdaSymbol{(}\AgdaBound{f'}\AgdaSpace{}%
\AgdaBound{x}\AgdaSymbol{))}\<%
\\
%
\>[2]\AgdaSymbol{→}\AgdaSpace{}%
\AgdaSymbol{(}\AgdaBound{x}\AgdaSpace{}%
\AgdaSymbol{:}\AgdaSpace{}%
\AgdaBound{A'}\AgdaSpace{}%
\AgdaOperator{\AgdaDatatype{∈}}\AgdaSpace{}%
\AgdaBound{Γ}\AgdaSpace{}%
\AgdaOperator{\AgdaInductiveConstructor{,}}\AgdaSpace{}%
\AgdaBound{A}\AgdaSymbol{)}\<%
\\
%
\>[2]\AgdaSymbol{→}\AgdaSpace{}%
\AgdaFunction{extb}\AgdaSpace{}%
\AgdaBound{fb'}\AgdaSpace{}%
\AgdaSymbol{(}\AgdaFunction{embedLookup}\AgdaSpace{}%
\AgdaBound{x}\AgdaSymbol{)}\AgdaSpace{}%
\AgdaOperator{\AgdaDatatype{≡}}\AgdaSpace{}%
\AgdaFunction{embedLookup}\AgdaSpace{}%
\AgdaSymbol{(}\AgdaFunction{ext}\AgdaSpace{}%
\AgdaBound{f'}\AgdaSpace{}%
\AgdaBound{x}\AgdaSymbol{)}\<%
\\
\>[0]\AgdaFunction{ext-embed-lookup}\AgdaSpace{}%
\AgdaBound{f'}\AgdaSpace{}%
\AgdaBound{fb'}\AgdaSpace{}%
\AgdaBound{h}\AgdaSpace{}%
\AgdaInductiveConstructor{Z}\AgdaSpace{}%
\AgdaSymbol{=}\AgdaSpace{}%
\AgdaInductiveConstructor{refl}\<%
\\
\>[0]\AgdaFunction{ext-embed-lookup}\AgdaSpace{}%
\AgdaBound{f'}\AgdaSpace{}%
\AgdaBound{fb'}\AgdaSpace{}%
\AgdaBound{h}\AgdaSpace{}%
\AgdaSymbol{(}\AgdaInductiveConstructor{S}\AgdaSpace{}%
\AgdaBound{x}\AgdaSymbol{)}\AgdaSpace{}%
\AgdaSymbol{=}\AgdaSpace{}%
\AgdaFunction{cong}\AgdaSpace{}%
\AgdaInductiveConstructor{Sb}\AgdaSpace{}%
\AgdaSymbol{(}\AgdaBound{h}\AgdaSpace{}%
\AgdaBound{x}\AgdaSymbol{)}\<%
\\
%
\\[\AgdaEmptyExtraSkip]%
\>[0]\AgdaFunction{pres-rename}\AgdaSpace{}%
\AgdaSymbol{:}\AgdaSpace{}%
\AgdaSymbol{∀}\AgdaSpace{}%
\AgdaSymbol{\{}\AgdaBound{Γ}\AgdaSpace{}%
\AgdaBound{A}\AgdaSpace{}%
\AgdaBound{Δ}\AgdaSymbol{\}}\<%
\\
\>[0][@{}l@{\AgdaIndent{0}}]%
\>[2]\AgdaSymbol{→}\AgdaSpace{}%
\AgdaSymbol{(}\AgdaBound{f'}\AgdaSpace{}%
\AgdaSymbol{:}\AgdaSpace{}%
\AgdaSymbol{∀}\AgdaSpace{}%
\AgdaSymbol{\{}\AgdaBound{A}\AgdaSymbol{\}}\AgdaSpace{}%
\AgdaSymbol{→}\AgdaSpace{}%
\AgdaBound{A}\AgdaSpace{}%
\AgdaOperator{\AgdaDatatype{∈}}\AgdaSpace{}%
\AgdaBound{Γ}\AgdaSpace{}%
\AgdaSymbol{→}\AgdaSpace{}%
\AgdaBound{A}\AgdaSpace{}%
\AgdaOperator{\AgdaDatatype{∈}}\AgdaSpace{}%
\AgdaBound{Δ}\AgdaSymbol{)}\<%
\\
%
\>[2]\AgdaSymbol{→}\AgdaSpace{}%
\AgdaSymbol{(}\AgdaBound{fb'}\AgdaSpace{}%
\AgdaSymbol{:}\AgdaSpace{}%
\AgdaSymbol{∀}\AgdaSpace{}%
\AgdaSymbol{\{}\AgdaBound{Ab}\AgdaSymbol{\}}\AgdaSpace{}%
\AgdaSymbol{→}\AgdaSpace{}%
\AgdaBound{Ab}\AgdaSpace{}%
\AgdaOperator{\AgdaDatatype{∈b}}\AgdaSpace{}%
\AgdaFunction{embedContext}\AgdaSpace{}%
\AgdaBound{Γ}\AgdaSpace{}%
\AgdaSymbol{→}\AgdaSpace{}%
\AgdaBound{Ab}\AgdaSpace{}%
\AgdaOperator{\AgdaDatatype{∈b}}\AgdaSpace{}%
\AgdaFunction{embedContext}\AgdaSpace{}%
\AgdaBound{Δ}\AgdaSymbol{)}\<%
\\
%
\>[2]\AgdaSymbol{→}\AgdaSpace{}%
\AgdaSymbol{(\{}\AgdaBound{A'}\AgdaSpace{}%
\AgdaSymbol{:}\AgdaSpace{}%
\AgdaDatatype{Type}\AgdaSymbol{\}}\AgdaSpace{}%
\AgdaSymbol{(}\AgdaBound{x}\AgdaSpace{}%
\AgdaSymbol{:}\AgdaSpace{}%
\AgdaBound{A'}\AgdaSpace{}%
\AgdaOperator{\AgdaDatatype{∈}}\AgdaSpace{}%
\AgdaBound{Γ}\AgdaSymbol{)}\AgdaSpace{}%
\AgdaSymbol{→}\AgdaSpace{}%
\AgdaBound{fb'}\AgdaSpace{}%
\AgdaSymbol{(}\AgdaFunction{embedLookup}\AgdaSpace{}%
\AgdaBound{x}\AgdaSymbol{)}\AgdaSpace{}%
\AgdaOperator{\AgdaDatatype{≡}}\AgdaSpace{}%
\AgdaFunction{embedLookup}\AgdaSpace{}%
\AgdaSymbol{(}\AgdaBound{f'}\AgdaSpace{}%
\AgdaBound{x}\AgdaSymbol{))}\<%
\\
%
\>[2]\AgdaSymbol{→}\AgdaSpace{}%
\AgdaSymbol{(}\AgdaBound{t}\AgdaSpace{}%
\AgdaSymbol{:}\AgdaSpace{}%
\AgdaBound{Γ}\AgdaSpace{}%
\AgdaOperator{\AgdaDatatype{⊢}}\AgdaSpace{}%
\AgdaBound{A}\AgdaSymbol{)}\<%
\\
%
\>[2]\AgdaSymbol{→}\AgdaSpace{}%
\AgdaFunction{boxCalc.rename}\AgdaSpace{}%
\AgdaBound{fb'}\AgdaSpace{}%
\AgdaSymbol{(}\AgdaFunction{embedTerm}\AgdaSpace{}%
\AgdaBound{t}\AgdaSymbol{)}\AgdaSpace{}%
\AgdaOperator{\AgdaDatatype{≡}}\AgdaSpace{}%
\AgdaFunction{embedTerm}\AgdaSpace{}%
\AgdaSymbol{(}\AgdaFunction{calc.rename}\AgdaSpace{}%
\AgdaBound{f'}\AgdaSpace{}%
\AgdaBound{t}\AgdaSymbol{)}\<%
\\
\>[0]\AgdaFunction{pres-rename}\AgdaSpace{}%
\AgdaBound{f'}\AgdaSpace{}%
\AgdaBound{fb'}\AgdaSpace{}%
\AgdaBound{h}\AgdaSpace{}%
\AgdaSymbol{(}\AgdaInductiveConstructor{`}\AgdaSpace{}%
\AgdaBound{x}\AgdaSymbol{)}\AgdaSpace{}%
\AgdaSymbol{=}\<%
\\
\>[0][@{}l@{\AgdaIndent{0}}]%
\>[2]\AgdaFunction{cong}\AgdaSpace{}%
\AgdaInductiveConstructor{ε}\AgdaSpace{}%
\AgdaSymbol{(}\AgdaBound{h}\AgdaSpace{}%
\AgdaBound{x}\AgdaSymbol{)}\<%
\\
\>[0]\AgdaFunction{pres-rename}\AgdaSpace{}%
\AgdaBound{f'}\AgdaSpace{}%
\AgdaBound{fb'}\AgdaSpace{}%
\AgdaBound{h}\AgdaSpace{}%
\AgdaSymbol{(}\AgdaInductiveConstructor{ƛ}\AgdaSpace{}%
\AgdaBound{t}\AgdaSymbol{)}\AgdaSpace{}%
\AgdaSymbol{=}\<%
\\
\>[0][@{}l@{\AgdaIndent{0}}]%
\>[2]\AgdaFunction{cong}\AgdaSpace{}%
\AgdaInductiveConstructor{ƛb}\AgdaSpace{}%
\AgdaSymbol{(}\AgdaFunction{pres-rename}\AgdaSpace{}%
\AgdaSymbol{(}\AgdaFunction{calc.ext}\AgdaSpace{}%
\AgdaBound{f'}\AgdaSymbol{)}\AgdaSpace{}%
\AgdaSymbol{(}\AgdaFunction{boxCalc.extb}\AgdaSpace{}%
\AgdaBound{fb'}\AgdaSymbol{)}\AgdaSpace{}%
\AgdaSymbol{(}\AgdaFunction{ext-embed-lookup}\AgdaSpace{}%
\AgdaBound{f'}\AgdaSpace{}%
\AgdaBound{fb'}\AgdaSpace{}%
\AgdaBound{h}\AgdaSymbol{)}\AgdaSpace{}%
\AgdaBound{t}\AgdaSymbol{)}\<%
\\
\>[0]\AgdaFunction{pres-rename}\AgdaSpace{}%
\AgdaBound{f'}\AgdaSpace{}%
\AgdaBound{fb'}\AgdaSpace{}%
\AgdaBound{h}\AgdaSpace{}%
\AgdaSymbol{(}\AgdaBound{t}\AgdaSpace{}%
\AgdaOperator{\AgdaInductiveConstructor{·}}\AgdaSpace{}%
\AgdaBound{t₁}\AgdaSymbol{)}\AgdaSpace{}%
\AgdaSymbol{=}\<%
\\
\>[0][@{}l@{\AgdaIndent{0}}]%
\>[2]\AgdaFunction{cong₂}\AgdaSpace{}%
\AgdaOperator{\AgdaInductiveConstructor{\AgdaUnderscore{}·b\AgdaUnderscore{}}}\AgdaSpace{}%
\AgdaSymbol{(}\AgdaFunction{pres-rename}\AgdaSpace{}%
\AgdaBound{f'}\AgdaSpace{}%
\AgdaBound{fb'}\AgdaSpace{}%
\AgdaBound{h}\AgdaSpace{}%
\AgdaBound{t}\AgdaSymbol{)}\AgdaSpace{}%
\AgdaSymbol{(}\AgdaFunction{cong}\AgdaSpace{}%
\AgdaInductiveConstructor{box}\AgdaSpace{}%
\AgdaSymbol{(}\AgdaFunction{pres-rename}\AgdaSpace{}%
\AgdaBound{f'}\AgdaSpace{}%
\AgdaBound{fb'}\AgdaSpace{}%
\AgdaBound{h}\AgdaSpace{}%
\AgdaBound{t₁}\AgdaSymbol{))}\<%
\\
%
\\[\AgdaEmptyExtraSkip]%
%
\\[\AgdaEmptyExtraSkip]%
\>[0]\AgdaFunction{exts-pres-rename}\AgdaSpace{}%
\AgdaSymbol{:}\AgdaSpace{}%
\AgdaSymbol{∀}\AgdaSpace{}%
\AgdaSymbol{\{}\AgdaBound{Γ}\AgdaSpace{}%
\AgdaBound{Δ}\AgdaSymbol{\}}\<%
\\
\>[0][@{}l@{\AgdaIndent{0}}]%
\>[2]\AgdaSymbol{→}\AgdaSpace{}%
\AgdaSymbol{(}\AgdaBound{fb'}\AgdaSpace{}%
\AgdaSymbol{:}\AgdaSpace{}%
\AgdaSymbol{∀}\AgdaSpace{}%
\AgdaSymbol{\{}\AgdaBound{Ab}\AgdaSymbol{\}}\AgdaSpace{}%
\AgdaSymbol{→}\AgdaSpace{}%
\AgdaInductiveConstructor{□}\AgdaSpace{}%
\AgdaBound{Ab}\AgdaSpace{}%
\AgdaOperator{\AgdaDatatype{∈b}}\AgdaSpace{}%
\AgdaFunction{embedContext}\AgdaSpace{}%
\AgdaBound{Γ}\AgdaSpace{}%
\AgdaSymbol{→}\AgdaSpace{}%
\AgdaFunction{embedContext}\AgdaSpace{}%
\AgdaBound{Δ}\AgdaSpace{}%
\AgdaOperator{\AgdaDatatype{⊢b}}\AgdaSpace{}%
\AgdaBound{Ab}\AgdaSymbol{)}\<%
\\
%
\>[2]\AgdaSymbol{→}\AgdaSpace{}%
\AgdaSymbol{(}\AgdaBound{f'}\AgdaSpace{}%
\AgdaSymbol{:}\AgdaSpace{}%
\AgdaSymbol{∀}\AgdaSpace{}%
\AgdaSymbol{\{}\AgdaBound{A}\AgdaSymbol{\}}\AgdaSpace{}%
\AgdaSymbol{→}\AgdaSpace{}%
\AgdaBound{A}\AgdaSpace{}%
\AgdaOperator{\AgdaDatatype{∈}}\AgdaSpace{}%
\AgdaBound{Γ}\AgdaSpace{}%
\AgdaSymbol{→}\AgdaSpace{}%
\AgdaBound{Δ}\AgdaSpace{}%
\AgdaOperator{\AgdaDatatype{⊢}}\AgdaSpace{}%
\AgdaBound{A}\AgdaSymbol{)}\<%
\\
%
\>[2]\AgdaSymbol{→}\AgdaSpace{}%
\AgdaSymbol{(\{}\AgdaBound{A'}\AgdaSpace{}%
\AgdaSymbol{:}\AgdaSpace{}%
\AgdaDatatype{Type}\AgdaSymbol{\}}\AgdaSpace{}%
\AgdaSymbol{(}\AgdaBound{x}\AgdaSpace{}%
\AgdaSymbol{:}\AgdaSpace{}%
\AgdaBound{A'}\AgdaSpace{}%
\AgdaOperator{\AgdaDatatype{∈}}\AgdaSpace{}%
\AgdaBound{Γ}\AgdaSymbol{)}\AgdaSpace{}%
\AgdaSymbol{→}\AgdaSpace{}%
\AgdaBound{fb'}\AgdaSpace{}%
\AgdaSymbol{(}\AgdaFunction{embedLookup}\AgdaSpace{}%
\AgdaBound{x}\AgdaSymbol{)}\AgdaSpace{}%
\AgdaOperator{\AgdaDatatype{≡}}\AgdaSpace{}%
\AgdaFunction{embedTerm}\AgdaSpace{}%
\AgdaSymbol{(}\AgdaBound{f'}\AgdaSpace{}%
\AgdaBound{x}\AgdaSymbol{))}\<%
\\
%
\>[2]\AgdaSymbol{→}\AgdaSpace{}%
\AgdaSymbol{\{}\AgdaBound{A'}\AgdaSpace{}%
\AgdaBound{A}\AgdaSpace{}%
\AgdaSymbol{:}\AgdaSpace{}%
\AgdaDatatype{Type}\AgdaSymbol{\}}\AgdaSpace{}%
\AgdaSymbol{(}\AgdaBound{x}\AgdaSpace{}%
\AgdaSymbol{:}\AgdaSpace{}%
\AgdaBound{A'}\AgdaSpace{}%
\AgdaOperator{\AgdaDatatype{∈}}\AgdaSpace{}%
\AgdaBound{Γ}\AgdaSpace{}%
\AgdaOperator{\AgdaInductiveConstructor{,}}\AgdaSpace{}%
\AgdaBound{A}\AgdaSymbol{)}\AgdaSpace{}%
\AgdaSymbol{→}\AgdaSpace{}%
\AgdaFunction{boxCalc.exts}\AgdaSpace{}%
\AgdaBound{fb'}\AgdaSpace{}%
\AgdaSymbol{(}\AgdaFunction{embedLookup}\AgdaSpace{}%
\AgdaBound{x}\AgdaSymbol{)}\AgdaSpace{}%
\AgdaOperator{\AgdaDatatype{≡}}\AgdaSpace{}%
\AgdaFunction{embedTerm}\AgdaSpace{}%
\AgdaSymbol{(}\AgdaFunction{calc.exts}\AgdaSpace{}%
\AgdaBound{f'}\AgdaSpace{}%
\AgdaBound{x}\AgdaSymbol{)}\<%
\\
\>[0]\AgdaFunction{exts-pres-rename}\AgdaSpace{}%
\AgdaBound{fb'}\AgdaSpace{}%
\AgdaBound{f'}\AgdaSpace{}%
\AgdaBound{h}\AgdaSpace{}%
\AgdaInductiveConstructor{Z}\AgdaSpace{}%
\AgdaSymbol{=}\AgdaSpace{}%
\AgdaInductiveConstructor{refl}\<%
\\
\>[0]\AgdaFunction{exts-pres-rename}\AgdaSpace{}%
\AgdaBound{fb'}\AgdaSpace{}%
\AgdaBound{f'}\AgdaSpace{}%
\AgdaBound{h}\AgdaSpace{}%
\AgdaSymbol{(}\AgdaInductiveConstructor{S}\AgdaSpace{}%
\AgdaBound{x}\AgdaSymbol{)}\AgdaSpace{}%
\AgdaSymbol{=}\<%
\\
\>[0][@{}l@{\AgdaIndent{0}}]%
\>[2]\AgdaOperator{\AgdaFunction{beginEq}}\<%
\\
\>[2][@{}l@{\AgdaIndent{0}}]%
\>[4]\AgdaFunction{boxCalc.rename}\AgdaSpace{}%
\AgdaInductiveConstructor{Sb}\AgdaSpace{}%
\AgdaSymbol{(}\AgdaBound{fb'}\AgdaSpace{}%
\AgdaSymbol{(}\AgdaFunction{embedLookup}\AgdaSpace{}%
\AgdaBound{x}\AgdaSymbol{))}\<%
\\
%
\>[2]\AgdaFunction{≡⟨}\AgdaSpace{}%
\AgdaFunction{cong}\AgdaSpace{}%
\AgdaSymbol{(}\AgdaFunction{boxCalc.rename}\AgdaSpace{}%
\AgdaInductiveConstructor{Sb}\AgdaSymbol{)}\AgdaSpace{}%
\AgdaSymbol{(}\AgdaBound{h}\AgdaSpace{}%
\AgdaBound{x}\AgdaSymbol{)}\AgdaSpace{}%
\AgdaFunction{⟩}\<%
\\
\>[2][@{}l@{\AgdaIndent{0}}]%
\>[4]\AgdaFunction{boxCalc.rename}\AgdaSpace{}%
\AgdaInductiveConstructor{Sb}\AgdaSpace{}%
\AgdaSymbol{(}\AgdaFunction{embedTerm}\AgdaSpace{}%
\AgdaSymbol{(}\AgdaBound{f'}\AgdaSpace{}%
\AgdaBound{x}\AgdaSymbol{))}\<%
\\
%
\>[2]\AgdaFunction{≡⟨}\AgdaSpace{}%
\AgdaFunction{pres-rename}\AgdaSpace{}%
\AgdaInductiveConstructor{S}\AgdaSpace{}%
\AgdaInductiveConstructor{Sb}\AgdaSpace{}%
\AgdaSymbol{(λ}\AgdaSpace{}%
\AgdaBound{x₁}\AgdaSpace{}%
\AgdaSymbol{→}\AgdaSpace{}%
\AgdaInductiveConstructor{refl}\AgdaSymbol{)}\AgdaSpace{}%
\AgdaSymbol{(}\AgdaBound{f'}\AgdaSpace{}%
\AgdaBound{x}\AgdaSymbol{)}\AgdaSpace{}%
\AgdaFunction{⟩}\<%
\\
\>[2][@{}l@{\AgdaIndent{0}}]%
\>[4]\AgdaFunction{embedTerm}\AgdaSpace{}%
\AgdaSymbol{(}\AgdaFunction{calc.rename}\AgdaSpace{}%
\AgdaInductiveConstructor{S}\AgdaSpace{}%
\AgdaSymbol{(}\AgdaBound{f'}\AgdaSpace{}%
\AgdaBound{x}\AgdaSymbol{))}\<%
\\
%
\>[2]\AgdaOperator{\AgdaFunction{∎eq}}\<%
\\
%
\\[\AgdaEmptyExtraSkip]%
\>[0]\AgdaFunction{pres-simultaneous-subst}\AgdaSpace{}%
\AgdaSymbol{:}\AgdaSpace{}%
\AgdaSymbol{∀}\AgdaSpace{}%
\AgdaSymbol{\{}\AgdaBound{Γ}\AgdaSpace{}%
\AgdaBound{Δ}\AgdaSpace{}%
\AgdaBound{A}\AgdaSymbol{\}}\<%
\\
\>[0][@{}l@{\AgdaIndent{0}}]%
\>[2]\AgdaSymbol{→}\AgdaSpace{}%
\AgdaSymbol{(}\AgdaBound{fb'}\AgdaSpace{}%
\AgdaSymbol{:}\AgdaSpace{}%
\AgdaSymbol{∀}\AgdaSpace{}%
\AgdaSymbol{\{}\AgdaBound{Ab}\AgdaSymbol{\}}\AgdaSpace{}%
\AgdaSymbol{→}\AgdaSpace{}%
\AgdaInductiveConstructor{□}\AgdaSpace{}%
\AgdaBound{Ab}\AgdaSpace{}%
\AgdaOperator{\AgdaDatatype{∈b}}\AgdaSpace{}%
\AgdaFunction{embedContext}\AgdaSpace{}%
\AgdaBound{Γ}\AgdaSpace{}%
\AgdaSymbol{→}\AgdaSpace{}%
\AgdaFunction{embedContext}\AgdaSpace{}%
\AgdaBound{Δ}\AgdaSpace{}%
\AgdaOperator{\AgdaDatatype{⊢b}}\AgdaSpace{}%
\AgdaBound{Ab}\AgdaSymbol{)}\<%
\\
%
\>[2]\AgdaSymbol{→}\AgdaSpace{}%
\AgdaSymbol{(}\AgdaBound{f'}\AgdaSpace{}%
\AgdaSymbol{:}\AgdaSpace{}%
\AgdaSymbol{∀}\AgdaSpace{}%
\AgdaSymbol{\{}\AgdaBound{A}\AgdaSymbol{\}}\AgdaSpace{}%
\AgdaSymbol{→}\AgdaSpace{}%
\AgdaBound{A}\AgdaSpace{}%
\AgdaOperator{\AgdaDatatype{∈}}\AgdaSpace{}%
\AgdaBound{Γ}\AgdaSpace{}%
\AgdaSymbol{→}\AgdaSpace{}%
\AgdaBound{Δ}\AgdaSpace{}%
\AgdaOperator{\AgdaDatatype{⊢}}\AgdaSpace{}%
\AgdaBound{A}\AgdaSymbol{)}\<%
\\
%
\>[2]\AgdaSymbol{→}\AgdaSpace{}%
\AgdaSymbol{(}\AgdaBound{h}\AgdaSpace{}%
\AgdaSymbol{:}\AgdaSpace{}%
\AgdaSymbol{∀}\AgdaSpace{}%
\AgdaSymbol{\{}\AgdaBound{A'}\AgdaSymbol{\}}\AgdaSpace{}%
\AgdaSymbol{→}\AgdaSpace{}%
\AgdaSymbol{(}\AgdaBound{x}\AgdaSpace{}%
\AgdaSymbol{:}\AgdaSpace{}%
\AgdaBound{A'}\AgdaSpace{}%
\AgdaOperator{\AgdaDatatype{∈}}\AgdaSpace{}%
\AgdaBound{Γ}\AgdaSymbol{)}\AgdaSpace{}%
\AgdaSymbol{→}\AgdaSpace{}%
\AgdaBound{fb'}\AgdaSpace{}%
\AgdaSymbol{(}\AgdaFunction{embedLookup}\AgdaSpace{}%
\AgdaBound{x}\AgdaSymbol{)}\AgdaSpace{}%
\AgdaOperator{\AgdaDatatype{≡}}\AgdaSpace{}%
\AgdaFunction{embedTerm}\AgdaSpace{}%
\AgdaSymbol{(}\AgdaBound{f'}\AgdaSpace{}%
\AgdaBound{x}\AgdaSymbol{)}\AgdaSpace{}%
\AgdaSymbol{)}\<%
\\
%
\>[2]\AgdaSymbol{→}\AgdaSpace{}%
\AgdaSymbol{(}\AgdaBound{M}\AgdaSpace{}%
\AgdaSymbol{:}\AgdaSpace{}%
\AgdaBound{Γ}\AgdaSpace{}%
\AgdaOperator{\AgdaDatatype{⊢}}\AgdaSpace{}%
\AgdaBound{A}\AgdaSymbol{)}\<%
\\
%
\>[2]\AgdaSymbol{→}\AgdaSpace{}%
\AgdaFunction{substb}\AgdaSpace{}%
\AgdaBound{fb'}\AgdaSpace{}%
\AgdaSymbol{(}\AgdaFunction{embedTerm}\AgdaSpace{}%
\AgdaBound{M}\AgdaSymbol{)}\AgdaSpace{}%
\AgdaOperator{\AgdaDatatype{≡}}\AgdaSpace{}%
\AgdaFunction{embedTerm}\AgdaSpace{}%
\AgdaSymbol{(}\AgdaFunction{subst}\AgdaSpace{}%
\AgdaBound{f'}\AgdaSpace{}%
\AgdaBound{M}\AgdaSymbol{)}\<%
\\
\>[0]\AgdaFunction{pres-simultaneous-subst}\AgdaSpace{}%
\AgdaBound{fb'}\AgdaSpace{}%
\AgdaBound{f'}\AgdaSpace{}%
\AgdaBound{h}\AgdaSpace{}%
\AgdaSymbol{(}\AgdaInductiveConstructor{`}\AgdaSpace{}%
\AgdaBound{x}\AgdaSymbol{)}\AgdaSpace{}%
\AgdaSymbol{=}\<%
\\
\>[0][@{}l@{\AgdaIndent{0}}]%
\>[2]\AgdaBound{h}\AgdaSpace{}%
\AgdaBound{x}\<%
\\
\>[0]\AgdaFunction{pres-simultaneous-subst}\AgdaSpace{}%
\AgdaBound{fb'}\AgdaSpace{}%
\AgdaBound{f'}\AgdaSpace{}%
\AgdaBound{h}\AgdaSpace{}%
\AgdaSymbol{(}\AgdaInductiveConstructor{ƛ}\AgdaSpace{}%
\AgdaBound{M}\AgdaSymbol{)}\AgdaSpace{}%
\AgdaSymbol{=}\<%
\\
\>[0][@{}l@{\AgdaIndent{0}}]%
\>[2]\AgdaFunction{cong}\AgdaSpace{}%
\AgdaInductiveConstructor{ƛb}\AgdaSpace{}%
\AgdaSymbol{(}\AgdaSpace{}%
\AgdaFunction{pres-simultaneous-subst}\AgdaSpace{}%
\AgdaSymbol{(}\AgdaFunction{boxCalc.exts}\AgdaSpace{}%
\AgdaBound{fb'}\AgdaSymbol{)}\AgdaSpace{}%
\AgdaSymbol{(}\AgdaFunction{calc.exts}\AgdaSpace{}%
\AgdaBound{f'}\AgdaSymbol{)}\AgdaSpace{}%
\AgdaSymbol{(}\AgdaFunction{exts-pres-rename}\AgdaSpace{}%
\AgdaBound{fb'}\AgdaSpace{}%
\AgdaBound{f'}\AgdaSpace{}%
\AgdaBound{h}\AgdaSymbol{)}\AgdaSpace{}%
\AgdaBound{M}\AgdaSpace{}%
\AgdaSymbol{)}\<%
\\
\>[0]\AgdaFunction{pres-simultaneous-subst}\AgdaSpace{}%
\AgdaBound{fb'}\AgdaSpace{}%
\AgdaBound{f'}\AgdaSpace{}%
\AgdaBound{h}\AgdaSpace{}%
\AgdaSymbol{(}\AgdaBound{M}\AgdaSpace{}%
\AgdaOperator{\AgdaInductiveConstructor{·}}\AgdaSpace{}%
\AgdaBound{M₁}\AgdaSymbol{)}\AgdaSpace{}%
\AgdaSymbol{=}\<%
\\
\>[0][@{}l@{\AgdaIndent{0}}]%
\>[2]\AgdaFunction{cong₂}\AgdaSpace{}%
\AgdaOperator{\AgdaInductiveConstructor{\AgdaUnderscore{}·b\AgdaUnderscore{}}}\AgdaSpace{}%
\AgdaSymbol{(}\AgdaFunction{pres-simultaneous-subst}\AgdaSpace{}%
\AgdaBound{fb'}\AgdaSpace{}%
\AgdaBound{f'}\AgdaSpace{}%
\AgdaBound{h}\AgdaSpace{}%
\AgdaBound{M}\AgdaSymbol{)}\AgdaSpace{}%
\AgdaSymbol{(}\AgdaFunction{cong}\AgdaSpace{}%
\AgdaInductiveConstructor{box}\AgdaSpace{}%
\AgdaSymbol{(}\AgdaFunction{pres-simultaneous-subst}\AgdaSpace{}%
\AgdaBound{fb'}\AgdaSpace{}%
\AgdaBound{f'}\AgdaSpace{}%
\AgdaBound{h}\AgdaSpace{}%
\AgdaBound{M₁}\AgdaSymbol{))}\<%
\\
%
\\[\AgdaEmptyExtraSkip]%
%
\\[\AgdaEmptyExtraSkip]%
\>[0]\AgdaComment{--\ Proof\ of\ h}\<%
\\
\>[0]\AgdaFunction{pres-single-subst}\AgdaSpace{}%
\AgdaSymbol{:}\AgdaSpace{}%
\AgdaSymbol{∀}\AgdaSpace{}%
\AgdaSymbol{\{}\AgdaBound{A}\AgdaSpace{}%
\AgdaBound{C}\AgdaSpace{}%
\AgdaBound{Γ}\AgdaSymbol{\}}\<%
\\
\>[0][@{}l@{\AgdaIndent{0}}]%
\>[2]\AgdaSymbol{→}\AgdaSpace{}%
\AgdaSymbol{(}\AgdaBound{N}\AgdaSpace{}%
\AgdaSymbol{:}\AgdaSpace{}%
\AgdaBound{Γ}\AgdaSpace{}%
\AgdaOperator{\AgdaDatatype{⊢}}\AgdaSpace{}%
\AgdaBound{C}\AgdaSymbol{)}\<%
\\
%
\>[2]\AgdaSymbol{→}\AgdaSpace{}%
\AgdaSymbol{(}\AgdaBound{x}\AgdaSpace{}%
\AgdaSymbol{:}\AgdaSpace{}%
\AgdaBound{A}\AgdaSpace{}%
\AgdaOperator{\AgdaDatatype{∈}}\AgdaSpace{}%
\AgdaBound{Γ}\AgdaSpace{}%
\AgdaOperator{\AgdaInductiveConstructor{,}}\AgdaSpace{}%
\AgdaBound{C}\AgdaSymbol{)}\<%
\\
%
\>[2]\AgdaSymbol{→}\AgdaSpace{}%
\AgdaFunction{fb}\AgdaSpace{}%
\AgdaSymbol{(}\AgdaFunction{embedTerm}\AgdaSpace{}%
\AgdaBound{N}\AgdaSymbol{)}\AgdaSpace{}%
\AgdaSymbol{(}\AgdaFunction{embedLookup}\AgdaSpace{}%
\AgdaBound{x}\AgdaSymbol{)}\AgdaSpace{}%
\AgdaOperator{\AgdaDatatype{≡}}\AgdaSpace{}%
\AgdaFunction{embedTerm}\AgdaSpace{}%
\AgdaSymbol{((}\AgdaFunction{f}\AgdaSpace{}%
\AgdaBound{N}\AgdaSymbol{)}\AgdaSpace{}%
\AgdaBound{x}\AgdaSymbol{)}\<%
\\
\>[0]\AgdaFunction{pres-single-subst}\AgdaSpace{}%
\AgdaBound{N}\AgdaSpace{}%
\AgdaInductiveConstructor{Z}\AgdaSpace{}%
\AgdaSymbol{=}\AgdaSpace{}%
\AgdaInductiveConstructor{refl}\<%
\\
\>[0]\AgdaFunction{pres-single-subst}\AgdaSpace{}%
\AgdaBound{N}\AgdaSpace{}%
\AgdaSymbol{(}\AgdaInductiveConstructor{S}\AgdaSpace{}%
\AgdaBound{x}\AgdaSymbol{)}\AgdaSpace{}%
\AgdaSymbol{=}\AgdaSpace{}%
\AgdaInductiveConstructor{refl}\<%
\\
%
\\[\AgdaEmptyExtraSkip]%
\>[0]\AgdaFunction{exts-pres-single-subst}\AgdaSpace{}%
\AgdaSymbol{:}\AgdaSpace{}%
\AgdaSymbol{∀}\AgdaSpace{}%
\AgdaSymbol{\{}\AgdaBound{A}\AgdaSpace{}%
\AgdaBound{C}\AgdaSpace{}%
\AgdaBound{Γ}\AgdaSpace{}%
\AgdaBound{B}\AgdaSymbol{\}}\<%
\\
\>[0][@{}l@{\AgdaIndent{0}}]%
\>[2]\AgdaSymbol{→}\AgdaSpace{}%
\AgdaSymbol{(}\AgdaBound{N}\AgdaSpace{}%
\AgdaSymbol{:}\AgdaSpace{}%
\AgdaBound{Γ}\AgdaSpace{}%
\AgdaOperator{\AgdaDatatype{⊢}}\AgdaSpace{}%
\AgdaBound{C}\AgdaSymbol{)}\<%
\\
%
\>[2]\AgdaSymbol{→}\AgdaSpace{}%
\AgdaSymbol{(}\AgdaBound{x}\AgdaSpace{}%
\AgdaSymbol{:}\AgdaSpace{}%
\AgdaBound{A}\AgdaSpace{}%
\AgdaOperator{\AgdaDatatype{∈}}\AgdaSpace{}%
\AgdaBound{Γ}\AgdaSpace{}%
\AgdaOperator{\AgdaInductiveConstructor{,}}\AgdaSpace{}%
\AgdaBound{C}\AgdaSpace{}%
\AgdaOperator{\AgdaInductiveConstructor{,}}\AgdaSpace{}%
\AgdaBound{B}\AgdaSymbol{)}\<%
\\
%
\>[2]\AgdaSymbol{→}\AgdaSpace{}%
\AgdaFunction{boxCalc.exts}\AgdaSpace{}%
\AgdaSymbol{(}\AgdaFunction{fb}\AgdaSpace{}%
\AgdaSymbol{(}\AgdaFunction{embedTerm}\AgdaSpace{}%
\AgdaBound{N}\AgdaSymbol{))}\AgdaSpace{}%
\AgdaSymbol{(}\AgdaFunction{embedLookup}\AgdaSpace{}%
\AgdaBound{x}\AgdaSymbol{)}\AgdaSpace{}%
\AgdaOperator{\AgdaDatatype{≡}}\AgdaSpace{}%
\AgdaFunction{embedTerm}\AgdaSpace{}%
\AgdaSymbol{(}\AgdaFunction{calc.exts}\AgdaSpace{}%
\AgdaSymbol{(}\AgdaFunction{f}\AgdaSpace{}%
\AgdaBound{N}\AgdaSymbol{)}\AgdaSpace{}%
\AgdaBound{x}\AgdaSymbol{)}\<%
\\
\>[0]\AgdaFunction{exts-pres-single-subst}\AgdaSpace{}%
\AgdaBound{N}\AgdaSpace{}%
\AgdaInductiveConstructor{Z}\AgdaSpace{}%
\AgdaSymbol{=}\AgdaSpace{}%
\AgdaInductiveConstructor{refl}\<%
\\
\>[0]\AgdaFunction{exts-pres-single-subst}\AgdaSpace{}%
\AgdaBound{N}\AgdaSpace{}%
\AgdaSymbol{(}\AgdaInductiveConstructor{S}\AgdaSpace{}%
\AgdaBound{x}\AgdaSymbol{)}\AgdaSpace{}%
\AgdaSymbol{=}\<%
\\
\>[0][@{}l@{\AgdaIndent{0}}]%
\>[2]\AgdaOperator{\AgdaFunction{beginEq}}\<%
\\
\>[2][@{}l@{\AgdaIndent{0}}]%
\>[4]\AgdaFunction{boxCalc.rename}\AgdaSpace{}%
\AgdaInductiveConstructor{Sb}\AgdaSpace{}%
\AgdaSymbol{(}\AgdaFunction{fb}\AgdaSpace{}%
\AgdaSymbol{(}\AgdaFunction{embedTerm}\AgdaSpace{}%
\AgdaBound{N}\AgdaSymbol{)}\AgdaSpace{}%
\AgdaSymbol{(}\AgdaFunction{embedLookup}\AgdaSpace{}%
\AgdaBound{x}\AgdaSymbol{))}\<%
\\
%
\>[2]\AgdaFunction{≡⟨}\AgdaSpace{}%
\AgdaFunction{cong}\AgdaSpace{}%
\AgdaSymbol{(}\AgdaFunction{boxCalc.rename}\AgdaSpace{}%
\AgdaInductiveConstructor{Sb}\AgdaSymbol{)}\AgdaSpace{}%
\AgdaSymbol{(}\AgdaFunction{pres-single-subst}\AgdaSpace{}%
\AgdaBound{N}\AgdaSpace{}%
\AgdaBound{x}\AgdaSymbol{)}\AgdaSpace{}%
\AgdaFunction{⟩}\<%
\\
\>[2][@{}l@{\AgdaIndent{0}}]%
\>[4]\AgdaFunction{boxCalc.rename}\AgdaSpace{}%
\AgdaInductiveConstructor{Sb}\AgdaSpace{}%
\AgdaSymbol{(}\AgdaFunction{embedTerm}\AgdaSpace{}%
\AgdaSymbol{((}\AgdaFunction{f}\AgdaSpace{}%
\AgdaBound{N}\AgdaSymbol{)}\AgdaSpace{}%
\AgdaBound{x}\AgdaSymbol{))}\<%
\\
%
\>[2]\AgdaFunction{≡⟨}\AgdaSpace{}%
\AgdaFunction{pres-rename}\AgdaSpace{}%
\AgdaInductiveConstructor{S}\AgdaSpace{}%
\AgdaInductiveConstructor{Sb}\AgdaSpace{}%
\AgdaSymbol{(λ}\AgdaSpace{}%
\AgdaBound{x₁}\AgdaSpace{}%
\AgdaSymbol{→}\AgdaSpace{}%
\AgdaInductiveConstructor{refl}\AgdaSymbol{)}\AgdaSpace{}%
\AgdaSymbol{(}\AgdaFunction{f}\AgdaSpace{}%
\AgdaBound{N}\AgdaSpace{}%
\AgdaBound{x}\AgdaSymbol{)}\AgdaSpace{}%
\AgdaFunction{⟩}\<%
\\
\>[2][@{}l@{\AgdaIndent{0}}]%
\>[4]\AgdaFunction{embedTerm}\AgdaSpace{}%
\AgdaSymbol{(}\AgdaFunction{calc.rename}\AgdaSpace{}%
\AgdaInductiveConstructor{S}\AgdaSpace{}%
\AgdaSymbol{(}\AgdaFunction{f}\AgdaSpace{}%
\AgdaBound{N}\AgdaSpace{}%
\AgdaBound{x}\AgdaSymbol{))}\<%
\\
%
\>[2]\AgdaOperator{\AgdaFunction{∎eq}}\<%
\\
%
\\[\AgdaEmptyExtraSkip]%
\>[0]\AgdaComment{--\ Reflection\ and\ pres\ of\ single\ substitution\ of\ refined\ Girard's\ translation\ }\<%
\\
\>[0]\AgdaFunction{pres-subst}\AgdaSpace{}%
\AgdaSymbol{:}\AgdaSpace{}%
\AgdaSymbol{∀}\AgdaSpace{}%
\AgdaSymbol{\{}\AgdaBound{Γ}\AgdaSpace{}%
\AgdaBound{A}\AgdaSpace{}%
\AgdaBound{B}\AgdaSymbol{\}}\AgdaSpace{}%
\AgdaSymbol{(}\AgdaBound{M}\AgdaSpace{}%
\AgdaSymbol{:}\AgdaSpace{}%
\AgdaBound{Γ}\AgdaSpace{}%
\AgdaOperator{\AgdaInductiveConstructor{,}}\AgdaSpace{}%
\AgdaBound{B}\AgdaSpace{}%
\AgdaOperator{\AgdaDatatype{⊢}}\AgdaSpace{}%
\AgdaBound{A}\AgdaSymbol{)}\AgdaSpace{}%
\AgdaSymbol{→}\AgdaSpace{}%
\AgdaSymbol{(}\AgdaBound{N}\AgdaSpace{}%
\AgdaSymbol{:}\AgdaSpace{}%
\AgdaBound{Γ}\AgdaSpace{}%
\AgdaOperator{\AgdaDatatype{⊢}}\AgdaSpace{}%
\AgdaBound{B}\AgdaSymbol{)}\AgdaSpace{}%
\AgdaSymbol{→}\AgdaSpace{}%
\AgdaSymbol{(}\AgdaFunction{embedTerm}\AgdaSpace{}%
\AgdaBound{M}\AgdaSymbol{)}\AgdaSpace{}%
\AgdaOperator{\AgdaFunction{[}}\AgdaSpace{}%
\AgdaFunction{embedTerm}\AgdaSpace{}%
\AgdaBound{N}\AgdaSpace{}%
\AgdaOperator{\AgdaFunction{]b}}\AgdaSpace{}%
\AgdaOperator{\AgdaDatatype{≡}}\AgdaSpace{}%
\AgdaFunction{embedTerm}\AgdaSpace{}%
\AgdaSymbol{(}\AgdaBound{M}\AgdaSpace{}%
\AgdaOperator{\AgdaFunction{[}}\AgdaSpace{}%
\AgdaBound{N}\AgdaSpace{}%
\AgdaOperator{\AgdaFunction{]}}\AgdaSymbol{)}\<%
\\
\>[0]\AgdaFunction{pres-subst}\AgdaSpace{}%
\AgdaSymbol{(}\AgdaInductiveConstructor{`}\AgdaSpace{}%
\AgdaInductiveConstructor{Z}\AgdaSymbol{)}\AgdaSpace{}%
\AgdaBound{N}\AgdaSpace{}%
\AgdaSymbol{=}\AgdaSpace{}%
\AgdaInductiveConstructor{refl}\<%
\\
\>[0]\AgdaFunction{pres-subst}\AgdaSpace{}%
\AgdaSymbol{(}\AgdaInductiveConstructor{`}\AgdaSpace{}%
\AgdaSymbol{(}\AgdaInductiveConstructor{S}\AgdaSpace{}%
\AgdaBound{x}\AgdaSymbol{))}\AgdaSpace{}%
\AgdaBound{N}\AgdaSpace{}%
\AgdaSymbol{=}\AgdaSpace{}%
\AgdaInductiveConstructor{refl}\<%
\\
\>[0]\AgdaFunction{pres-subst}\AgdaSpace{}%
\AgdaSymbol{(}\AgdaInductiveConstructor{ƛ}\AgdaSpace{}%
\AgdaBound{M}\AgdaSymbol{)}\AgdaSpace{}%
\AgdaBound{N}\AgdaSpace{}%
\AgdaSymbol{=}\<%
\\
\>[0][@{}l@{\AgdaIndent{0}}]%
\>[2]\AgdaOperator{\AgdaFunction{beginEq}}\<%
\\
\>[2][@{}l@{\AgdaIndent{0}}]%
\>[4]\AgdaSymbol{(}\AgdaFunction{embedTerm}\AgdaSpace{}%
\AgdaSymbol{(}\AgdaInductiveConstructor{ƛ}\AgdaSpace{}%
\AgdaBound{M}\AgdaSymbol{))}\AgdaSpace{}%
\AgdaOperator{\AgdaFunction{[}}\AgdaSpace{}%
\AgdaFunction{embedTerm}\AgdaSpace{}%
\AgdaBound{N}\AgdaSpace{}%
\AgdaOperator{\AgdaFunction{]b}}\<%
\\
%
\>[2]\AgdaFunction{≡⟨}\AgdaSpace{}%
\AgdaInductiveConstructor{refl}\AgdaSpace{}%
\AgdaFunction{⟩}\<%
\\
\>[2][@{}l@{\AgdaIndent{0}}]%
\>[4]\AgdaInductiveConstructor{ƛb}\AgdaSpace{}%
\AgdaSymbol{(}\AgdaFunction{substb}\AgdaSpace{}%
\AgdaSymbol{(}\AgdaFunction{boxCalc.exts}\AgdaSpace{}%
\AgdaSymbol{(}\AgdaFunction{fb}\AgdaSpace{}%
\AgdaSymbol{(}\AgdaFunction{embedTerm}\AgdaSpace{}%
\AgdaBound{N}\AgdaSymbol{)))}\AgdaSpace{}%
\AgdaSymbol{(}\AgdaFunction{embedTerm}\AgdaSpace{}%
\AgdaBound{M}\AgdaSymbol{))}\<%
\\
%
\>[2]\AgdaFunction{≡⟨}\AgdaSpace{}%
\AgdaFunction{cong}\AgdaSpace{}%
\AgdaInductiveConstructor{ƛb}\AgdaSpace{}%
\AgdaSymbol{(}\AgdaFunction{pres-simultaneous-subst}\AgdaSpace{}%
\AgdaSymbol{(}\AgdaFunction{boxCalc.exts}\AgdaSpace{}%
\AgdaSymbol{(}\AgdaFunction{fb}\AgdaSpace{}%
\AgdaSymbol{(}\AgdaFunction{embedTerm}\AgdaSpace{}%
\AgdaBound{N}\AgdaSymbol{)))}\AgdaSpace{}%
\AgdaSymbol{(}\AgdaFunction{calc.exts}\AgdaSpace{}%
\AgdaSymbol{(}\AgdaFunction{f}\AgdaSpace{}%
\AgdaBound{N}\AgdaSymbol{))}\AgdaSpace{}%
\AgdaSymbol{(}\AgdaFunction{exts-pres-single-subst}\AgdaSpace{}%
\AgdaBound{N}\AgdaSymbol{)}\AgdaSpace{}%
\AgdaBound{M}\AgdaSymbol{)}\AgdaSpace{}%
\AgdaFunction{⟩}\<%
\\
\>[2][@{}l@{\AgdaIndent{0}}]%
\>[4]\AgdaInductiveConstructor{ƛb}\AgdaSpace{}%
\AgdaSymbol{(}\AgdaFunction{embedTerm}\AgdaSpace{}%
\AgdaSymbol{(}\AgdaFunction{subst}\AgdaSpace{}%
\AgdaSymbol{(}\AgdaFunction{calc.exts}\AgdaSpace{}%
\AgdaSymbol{\AgdaUnderscore{})}\AgdaSpace{}%
\AgdaBound{M}\AgdaSymbol{))}\<%
\\
%
\>[2]\AgdaFunction{≡⟨}\AgdaSpace{}%
\AgdaInductiveConstructor{refl}\AgdaSpace{}%
\AgdaFunction{⟩}\<%
\\
\>[2][@{}l@{\AgdaIndent{0}}]%
\>[4]\AgdaFunction{embedTerm}\AgdaSpace{}%
\AgdaSymbol{(}\AgdaInductiveConstructor{ƛ}\AgdaSpace{}%
\AgdaBound{M}\AgdaSpace{}%
\AgdaOperator{\AgdaFunction{[}}\AgdaSpace{}%
\AgdaBound{N}\AgdaSpace{}%
\AgdaOperator{\AgdaFunction{]}}\AgdaSymbol{)}\<%
\\
%
\>[2]\AgdaOperator{\AgdaFunction{∎eq}}\<%
\\
\>[0]\AgdaFunction{pres-subst}\AgdaSpace{}%
\AgdaSymbol{(}\AgdaBound{M}\AgdaSpace{}%
\AgdaOperator{\AgdaInductiveConstructor{·}}\AgdaSpace{}%
\AgdaBound{M₁}\AgdaSymbol{)}\AgdaSpace{}%
\AgdaBound{N}\AgdaSpace{}%
\AgdaSymbol{=}\<%
\\
\>[0][@{}l@{\AgdaIndent{0}}]%
\>[2]\AgdaOperator{\AgdaFunction{beginEq}}\<%
\\
\>[2][@{}l@{\AgdaIndent{0}}]%
\>[4]\AgdaSymbol{(}\AgdaFunction{embedTerm}\AgdaSpace{}%
\AgdaSymbol{(}\AgdaBound{M}\AgdaSpace{}%
\AgdaOperator{\AgdaInductiveConstructor{·}}\AgdaSpace{}%
\AgdaBound{M₁}\AgdaSymbol{))}\AgdaSpace{}%
\AgdaOperator{\AgdaFunction{[}}\AgdaSpace{}%
\AgdaFunction{embedTerm}\AgdaSpace{}%
\AgdaBound{N}\AgdaSpace{}%
\AgdaOperator{\AgdaFunction{]b}}\<%
\\
%
\>[2]\AgdaFunction{≡⟨}\AgdaSpace{}%
\AgdaInductiveConstructor{refl}\AgdaSpace{}%
\AgdaFunction{⟩}\<%
\\
\>[2][@{}l@{\AgdaIndent{0}}]%
\>[4]\AgdaSymbol{(}\AgdaFunction{embedTerm}\AgdaSpace{}%
\AgdaBound{M}\AgdaSymbol{)}\AgdaSpace{}%
\AgdaOperator{\AgdaFunction{[}}\AgdaSpace{}%
\AgdaFunction{embedTerm}\AgdaSpace{}%
\AgdaBound{N}\AgdaSpace{}%
\AgdaOperator{\AgdaFunction{]b}}\AgdaSpace{}%
\AgdaOperator{\AgdaInductiveConstructor{·b}}\AgdaSpace{}%
\AgdaInductiveConstructor{box}\AgdaSpace{}%
\AgdaSymbol{(}\AgdaFunction{embedTerm}\AgdaSpace{}%
\AgdaBound{M₁}\AgdaSymbol{)}\AgdaSpace{}%
\AgdaOperator{\AgdaFunction{[}}\AgdaSpace{}%
\AgdaFunction{embedTerm}\AgdaSpace{}%
\AgdaBound{N}\AgdaSpace{}%
\AgdaOperator{\AgdaFunction{]b}}\<%
\\
%
\>[2]\AgdaFunction{≡⟨}\AgdaSpace{}%
\AgdaFunction{cong}\AgdaSpace{}%
\AgdaSymbol{(}\AgdaOperator{\AgdaInductiveConstructor{\AgdaUnderscore{}·b}}\AgdaSpace{}%
\AgdaInductiveConstructor{box}\AgdaSpace{}%
\AgdaSymbol{(}\AgdaFunction{embedTerm}\AgdaSpace{}%
\AgdaBound{M₁}\AgdaSymbol{)}\AgdaSpace{}%
\AgdaOperator{\AgdaFunction{[}}\AgdaSpace{}%
\AgdaFunction{embedTerm}\AgdaSpace{}%
\AgdaBound{N}\AgdaSpace{}%
\AgdaOperator{\AgdaFunction{]b}}\AgdaSymbol{)}\AgdaSpace{}%
\AgdaSymbol{(}\AgdaFunction{pres-subst}\AgdaSpace{}%
\AgdaBound{M}\AgdaSpace{}%
\AgdaBound{N}\AgdaSymbol{)}\AgdaSpace{}%
\AgdaFunction{⟩}\<%
\\
\>[2][@{}l@{\AgdaIndent{0}}]%
\>[4]\AgdaFunction{embedTerm}\AgdaSpace{}%
\AgdaSymbol{(}\AgdaBound{M}\AgdaSpace{}%
\AgdaOperator{\AgdaFunction{[}}\AgdaSpace{}%
\AgdaBound{N}\AgdaSpace{}%
\AgdaOperator{\AgdaFunction{]}}\AgdaSymbol{)}\AgdaSpace{}%
\AgdaOperator{\AgdaInductiveConstructor{·b}}\AgdaSpace{}%
\AgdaInductiveConstructor{box}\AgdaSpace{}%
\AgdaSymbol{(}\AgdaFunction{embedTerm}\AgdaSpace{}%
\AgdaBound{M₁}\AgdaSymbol{)}\AgdaSpace{}%
\AgdaOperator{\AgdaFunction{[}}\AgdaSpace{}%
\AgdaFunction{embedTerm}\AgdaSpace{}%
\AgdaBound{N}\AgdaSpace{}%
\AgdaOperator{\AgdaFunction{]b}}\<%
\\
%
\>[2]\AgdaFunction{≡⟨}\AgdaSpace{}%
\AgdaFunction{cong}\AgdaSpace{}%
\AgdaSymbol{(λ}\AgdaSpace{}%
\AgdaBound{x}\AgdaSpace{}%
\AgdaSymbol{→}\AgdaSpace{}%
\AgdaFunction{embedTerm}\AgdaSpace{}%
\AgdaSymbol{(}\AgdaBound{M}\AgdaSpace{}%
\AgdaOperator{\AgdaFunction{[}}\AgdaSpace{}%
\AgdaBound{N}\AgdaSpace{}%
\AgdaOperator{\AgdaFunction{]}}\AgdaSymbol{)}\AgdaSpace{}%
\AgdaOperator{\AgdaInductiveConstructor{·b}}\AgdaSpace{}%
\AgdaBound{x}\AgdaSymbol{)}\AgdaSpace{}%
\AgdaSymbol{(}\AgdaFunction{cong}\AgdaSpace{}%
\AgdaInductiveConstructor{box}\AgdaSpace{}%
\AgdaSymbol{(}\AgdaFunction{pres-subst}\AgdaSpace{}%
\AgdaBound{M₁}\AgdaSpace{}%
\AgdaBound{N}\AgdaSymbol{))}\AgdaSpace{}%
\AgdaFunction{⟩}\<%
\\
\>[2][@{}l@{\AgdaIndent{0}}]%
\>[4]\AgdaFunction{embedTerm}\AgdaSpace{}%
\AgdaSymbol{(}\AgdaBound{M}\AgdaSpace{}%
\AgdaOperator{\AgdaFunction{[}}\AgdaSpace{}%
\AgdaBound{N}\AgdaSpace{}%
\AgdaOperator{\AgdaFunction{]}}\AgdaSymbol{)}\AgdaSpace{}%
\AgdaOperator{\AgdaInductiveConstructor{·b}}\AgdaSpace{}%
\AgdaInductiveConstructor{box}\AgdaSpace{}%
\AgdaSymbol{(}\AgdaFunction{embedTerm}\AgdaSpace{}%
\AgdaSymbol{(}\AgdaBound{M₁}\AgdaSpace{}%
\AgdaOperator{\AgdaFunction{[}}\AgdaSpace{}%
\AgdaBound{N}\AgdaSpace{}%
\AgdaOperator{\AgdaFunction{]}}\AgdaSymbol{))}\<%
\\
%
\>[2]\AgdaFunction{≡⟨}\AgdaSpace{}%
\AgdaInductiveConstructor{refl}\AgdaSpace{}%
\AgdaFunction{⟩}\<%
\\
\>[2][@{}l@{\AgdaIndent{0}}]%
\>[4]\AgdaFunction{embedTerm}\AgdaSpace{}%
\AgdaSymbol{((}\AgdaBound{M}\AgdaSpace{}%
\AgdaOperator{\AgdaInductiveConstructor{·}}\AgdaSpace{}%
\AgdaBound{M₁}\AgdaSymbol{)}\AgdaSpace{}%
\AgdaOperator{\AgdaFunction{[}}\AgdaSpace{}%
\AgdaBound{N}\AgdaSpace{}%
\AgdaOperator{\AgdaFunction{]}}\AgdaSymbol{)}\<%
\\
%
\>[2]\AgdaOperator{\AgdaFunction{∎eq}}\<%
\\
%
\\[\AgdaEmptyExtraSkip]%
\>[0]\AgdaComment{--\ Preservation\ of\ reduction\ of\ Girard's\ translation}\<%
\end{code}
\begin{code}%
\>[0]\AgdaFunction{pres-red}\AgdaSpace{}%
\AgdaSymbol{:}\AgdaSpace{}%
\AgdaSymbol{∀}\AgdaSpace{}%
\AgdaSymbol{\{}\AgdaBound{Γ}\AgdaSpace{}%
\AgdaBound{A}\AgdaSymbol{\}}\<%
\\
\>[0][@{}l@{\AgdaIndent{0}}]%
\>[2]\AgdaSymbol{→}\AgdaSpace{}%
\AgdaSymbol{(}\AgdaBound{M}\AgdaSpace{}%
\AgdaSymbol{:}\AgdaSpace{}%
\AgdaBound{Γ}\AgdaSpace{}%
\AgdaOperator{\AgdaDatatype{⊢}}\AgdaSpace{}%
\AgdaBound{A}\AgdaSymbol{)}\<%
\\
%
\>[2]\AgdaSymbol{→}\AgdaSpace{}%
\AgdaSymbol{(}\AgdaBound{N}\AgdaSpace{}%
\AgdaSymbol{:}\AgdaSpace{}%
\AgdaBound{Γ}\AgdaSpace{}%
\AgdaOperator{\AgdaDatatype{⊢}}\AgdaSpace{}%
\AgdaBound{A}\AgdaSymbol{)}\<%
\\
%
\>[2]\AgdaSymbol{→}\AgdaSpace{}%
\AgdaSymbol{(}\AgdaBound{M}\AgdaSpace{}%
\AgdaOperator{\AgdaDatatype{↝βₙ}}\AgdaSpace{}%
\AgdaBound{N}\AgdaSymbol{)}\<%
\\
%
\>[2]\AgdaSymbol{→}\AgdaSpace{}%
\AgdaSymbol{(}\AgdaFunction{embedTerm}\AgdaSpace{}%
\AgdaBound{M}\AgdaSpace{}%
\AgdaOperator{\AgdaDatatype{↝βb}}\AgdaSpace{}%
\AgdaFunction{embedTerm}\AgdaSpace{}%
\AgdaBound{N}\AgdaSymbol{)}\<%
\\
\>[0]\AgdaFunction{pres-red}\AgdaSpace{}%
\AgdaSymbol{(}\AgdaInductiveConstructor{ƛ}\AgdaSpace{}%
\AgdaBound{M₁}\AgdaSpace{}%
\AgdaOperator{\AgdaInductiveConstructor{·}}\AgdaSpace{}%
\AgdaBound{M₂}\AgdaSymbol{)}\AgdaSpace{}%
\AgdaBound{N}\AgdaSpace{}%
\AgdaInductiveConstructor{βₙ}\AgdaSpace{}%
\AgdaSymbol{=}\<%
\\
\>[0][@{}l@{\AgdaIndent{0}}]%
\>[2]\AgdaFunction{Eq.subst}\AgdaSpace{}%
\AgdaSymbol{(λ}\AgdaSpace{}%
\AgdaBound{t}\AgdaSpace{}%
\AgdaSymbol{→}\AgdaSpace{}%
\AgdaInductiveConstructor{ƛb}\AgdaSpace{}%
\AgdaSymbol{(}\AgdaFunction{embedTerm}\AgdaSpace{}%
\AgdaBound{M₁}\AgdaSymbol{)}\AgdaSpace{}%
\AgdaOperator{\AgdaInductiveConstructor{·b}}\AgdaSpace{}%
\AgdaInductiveConstructor{box}\AgdaSpace{}%
\AgdaSymbol{(}\AgdaFunction{embedTerm}\AgdaSpace{}%
\AgdaBound{M₂}\AgdaSymbol{)}\AgdaSpace{}%
\AgdaOperator{\AgdaDatatype{↝βb}}\AgdaSpace{}%
\AgdaBound{t}\AgdaSymbol{)}\<%
\\
%
\>[2]\AgdaSymbol{(}\AgdaFunction{pres-subst}\AgdaSpace{}%
\AgdaBound{M₁}\AgdaSpace{}%
\AgdaBound{M₂}\AgdaSymbol{)}\<%
\\
%
\>[2]\AgdaInductiveConstructor{βb}\<%
\\
\>[0]\AgdaFunction{pres-red}\AgdaSpace{}%
\AgdaBound{M}\AgdaSpace{}%
\AgdaBound{N}\AgdaSpace{}%
\AgdaSymbol{(}\AgdaInductiveConstructor{μ}\AgdaSpace{}%
\AgdaSymbol{\{}\AgdaArgument{M}\AgdaSpace{}%
\AgdaSymbol{=}\AgdaSpace{}%
\AgdaBound{M₁}\AgdaSymbol{\}}\AgdaSpace{}%
\AgdaSymbol{\{}\AgdaArgument{M'}\AgdaSpace{}%
\AgdaSymbol{=}\AgdaSpace{}%
\AgdaBound{M'}\AgdaSymbol{\}}\AgdaSpace{}%
\AgdaBound{red}\AgdaSymbol{)}\AgdaSpace{}%
\AgdaSymbol{=}\<%
\\
\>[0][@{}l@{\AgdaIndent{0}}]%
\>[2]\AgdaInductiveConstructor{μ}\AgdaSpace{}%
\AgdaSymbol{(}\AgdaFunction{pres-red}\AgdaSpace{}%
\AgdaBound{M₁}\AgdaSpace{}%
\AgdaBound{M'}\AgdaSpace{}%
\AgdaBound{red}\AgdaSymbol{)}\<%
\\
\>[0]\AgdaFunction{pres-red}\AgdaSpace{}%
\AgdaBound{M}\AgdaSpace{}%
\AgdaBound{N}\AgdaSpace{}%
\AgdaSymbol{(}\AgdaInductiveConstructor{ν<}\AgdaSpace{}%
\AgdaSymbol{\{}\AgdaArgument{M}\AgdaSpace{}%
\AgdaSymbol{=}\AgdaSpace{}%
\AgdaBound{M₁}\AgdaSymbol{\}}\AgdaSpace{}%
\AgdaSymbol{\{}\AgdaArgument{N}\AgdaSpace{}%
\AgdaSymbol{=}\AgdaSpace{}%
\AgdaBound{N₁}\AgdaSymbol{\}}\AgdaSpace{}%
\AgdaSymbol{\{}\AgdaArgument{N'}\AgdaSpace{}%
\AgdaSymbol{=}\AgdaSpace{}%
\AgdaBound{N'}\AgdaSymbol{\}}\AgdaSpace{}%
\AgdaBound{x}\AgdaSpace{}%
\AgdaBound{red}\AgdaSymbol{)}\AgdaSpace{}%
\AgdaSymbol{=}\<%
\\
\>[0][@{}l@{\AgdaIndent{0}}]%
\>[2]\AgdaInductiveConstructor{ν}\AgdaSpace{}%
\AgdaSymbol{(}\AgdaInductiveConstructor{ζ}\AgdaSpace{}%
\AgdaSymbol{(}\AgdaFunction{pres-red}\AgdaSpace{}%
\AgdaBound{N₁}\AgdaSpace{}%
\AgdaBound{N'}\AgdaSpace{}%
\AgdaBound{red}\AgdaSymbol{))}\<%
\\
\>[0]\AgdaFunction{pres-red}\AgdaSpace{}%
\AgdaBound{M}\AgdaSpace{}%
\AgdaBound{N}\AgdaSpace{}%
\AgdaSymbol{(}\AgdaInductiveConstructor{ν}\AgdaSpace{}%
\AgdaSymbol{\{}\AgdaArgument{M}\AgdaSpace{}%
\AgdaSymbol{=}\AgdaSpace{}%
\AgdaBound{M₁}\AgdaSymbol{\}}\AgdaSpace{}%
\AgdaSymbol{\{}\AgdaArgument{N}\AgdaSpace{}%
\AgdaSymbol{=}\AgdaSpace{}%
\AgdaBound{N₁}\AgdaSymbol{\}}\AgdaSpace{}%
\AgdaSymbol{\{}\AgdaArgument{N'}\AgdaSpace{}%
\AgdaSymbol{=}\AgdaSpace{}%
\AgdaBound{N'}\AgdaSymbol{\}}\AgdaSpace{}%
\AgdaBound{red}\AgdaSymbol{)}\AgdaSpace{}%
\AgdaSymbol{=}\<%
\\
\>[0][@{}l@{\AgdaIndent{0}}]%
\>[2]\AgdaInductiveConstructor{ν}\AgdaSpace{}%
\AgdaSymbol{(}\AgdaInductiveConstructor{ζ}\AgdaSpace{}%
\AgdaSymbol{(}\AgdaFunction{pres-red}\AgdaSpace{}%
\AgdaBound{N₁}\AgdaSpace{}%
\AgdaBound{N'}\AgdaSpace{}%
\AgdaBound{red}\AgdaSymbol{))}\<%
\\
\>[0]\AgdaFunction{pres-red}\AgdaSpace{}%
\AgdaBound{M}\AgdaSpace{}%
\AgdaBound{N}\AgdaSpace{}%
\AgdaSymbol{(}\AgdaInductiveConstructor{ξ}\AgdaSpace{}%
\AgdaSymbol{\{}\AgdaArgument{M}\AgdaSpace{}%
\AgdaSymbol{=}\AgdaSpace{}%
\AgdaBound{M₁}\AgdaSymbol{\}}\AgdaSpace{}%
\AgdaSymbol{\{}\AgdaArgument{M'}\AgdaSpace{}%
\AgdaSymbol{=}\AgdaSpace{}%
\AgdaBound{M'}\AgdaSymbol{\}}\AgdaSpace{}%
\AgdaBound{red}\AgdaSymbol{)}\AgdaSpace{}%
\AgdaSymbol{=}\<%
\\
\>[0][@{}l@{\AgdaIndent{0}}]%
\>[2]\AgdaInductiveConstructor{ξ}\AgdaSpace{}%
\AgdaSymbol{(}\AgdaFunction{pres-red}\AgdaSpace{}%
\AgdaBound{M₁}\AgdaSpace{}%
\AgdaBound{M'}\AgdaSpace{}%
\AgdaBound{red}\AgdaSymbol{)}\<%
\\
\>[0]\<%
\end{code}
\begin{code}[hide]%
\>[0]\AgdaFunction{pres-eval}\AgdaSpace{}%
\AgdaSymbol{:}\AgdaSpace{}%
\AgdaSymbol{∀}\AgdaSpace{}%
\AgdaSymbol{\{}\AgdaBound{Γ}\AgdaSpace{}%
\AgdaBound{A}\AgdaSymbol{\}}\<%
\\
\>[0][@{}l@{\AgdaIndent{0}}]%
\>[2]\AgdaSymbol{→}\AgdaSpace{}%
\AgdaSymbol{(}\AgdaBound{M}\AgdaSpace{}%
\AgdaSymbol{:}\AgdaSpace{}%
\AgdaBound{Γ}\AgdaSpace{}%
\AgdaOperator{\AgdaDatatype{⊢}}\AgdaSpace{}%
\AgdaBound{A}\AgdaSymbol{)}\<%
\\
%
\>[2]\AgdaSymbol{→}\AgdaSpace{}%
\AgdaSymbol{(}\AgdaBound{N}\AgdaSpace{}%
\AgdaSymbol{:}\AgdaSpace{}%
\AgdaBound{Γ}\AgdaSpace{}%
\AgdaOperator{\AgdaDatatype{⊢}}\AgdaSpace{}%
\AgdaBound{A}\AgdaSymbol{)}\<%
\\
%
\>[2]\AgdaSymbol{→}\AgdaSpace{}%
\AgdaSymbol{(}\AgdaBound{M}\AgdaSpace{}%
\AgdaOperator{\AgdaDatatype{↝ₙ}}\AgdaSpace{}%
\AgdaBound{N}\AgdaSymbol{)}\<%
\\
%
\>[2]\AgdaSymbol{→}\AgdaSpace{}%
\AgdaSymbol{(}\AgdaFunction{embedTerm}\AgdaSpace{}%
\AgdaBound{M}\AgdaSpace{}%
\AgdaOperator{\AgdaDatatype{↝bₙ}}\AgdaSpace{}%
\AgdaFunction{embedTerm}\AgdaSpace{}%
\AgdaBound{N}\AgdaSymbol{)}\<%
\\
\>[0]\AgdaFunction{pres-eval}\AgdaSpace{}%
\AgdaSymbol{(}\AgdaInductiveConstructor{ƛ}\AgdaSpace{}%
\AgdaBound{M₁}\AgdaSpace{}%
\AgdaOperator{\AgdaInductiveConstructor{·}}\AgdaSpace{}%
\AgdaBound{M₂}\AgdaSymbol{)}\AgdaSpace{}%
\AgdaBound{N}\AgdaSpace{}%
\AgdaInductiveConstructor{βₙ}\AgdaSpace{}%
\AgdaSymbol{=}\<%
\\
\>[0][@{}l@{\AgdaIndent{0}}]%
\>[2]\AgdaFunction{Eq.subst}\<%
\\
\>[2][@{}l@{\AgdaIndent{0}}]%
\>[4]\AgdaSymbol{(λ}\AgdaSpace{}%
\AgdaBound{t}\AgdaSpace{}%
\AgdaSymbol{→}\AgdaSpace{}%
\AgdaInductiveConstructor{ƛb}\AgdaSpace{}%
\AgdaSymbol{(}\AgdaFunction{embedTerm}\AgdaSpace{}%
\AgdaBound{M₁}\AgdaSymbol{)}\AgdaSpace{}%
\AgdaOperator{\AgdaInductiveConstructor{·b}}\AgdaSpace{}%
\AgdaInductiveConstructor{box}\AgdaSpace{}%
\AgdaSymbol{(}\AgdaFunction{embedTerm}\AgdaSpace{}%
\AgdaBound{M₂}\AgdaSymbol{)}\AgdaSpace{}%
\AgdaOperator{\AgdaDatatype{↝bₙ}}\AgdaSpace{}%
\AgdaBound{t}\AgdaSymbol{)}\<%
\\
%
\>[4]\AgdaSymbol{(}\AgdaFunction{pres-subst}\AgdaSpace{}%
\AgdaBound{M₁}\AgdaSpace{}%
\AgdaBound{M₂}\AgdaSymbol{)}\<%
\\
%
\>[4]\AgdaInductiveConstructor{βb}\<%
\\
\>[0]\AgdaFunction{pres-eval}\AgdaSpace{}%
\AgdaBound{M}\AgdaSpace{}%
\AgdaBound{N}\AgdaSpace{}%
\AgdaSymbol{(}\AgdaInductiveConstructor{μ}\AgdaSpace{}%
\AgdaSymbol{\{}\AgdaArgument{M}\AgdaSpace{}%
\AgdaSymbol{=}\AgdaSpace{}%
\AgdaBound{M₁}\AgdaSymbol{\}}\AgdaSpace{}%
\AgdaSymbol{\{}\AgdaArgument{M'}\AgdaSpace{}%
\AgdaSymbol{=}\AgdaSpace{}%
\AgdaBound{M'}\AgdaSymbol{\}}\AgdaSpace{}%
\AgdaBound{red}\AgdaSymbol{)}\AgdaSpace{}%
\AgdaSymbol{=}\<%
\\
\>[0][@{}l@{\AgdaIndent{0}}]%
\>[2]\AgdaInductiveConstructor{μ}\AgdaSpace{}%
\AgdaSymbol{(}\AgdaFunction{pres-eval}\AgdaSpace{}%
\AgdaBound{M₁}\AgdaSpace{}%
\AgdaBound{M'}\AgdaSpace{}%
\AgdaBound{red}\AgdaSymbol{)}\<%
\end{code}
  %   \end{block}
  % \end{frame}
  % \begin{frame}[fragile]{Research Results}
  %   \begin{block}{Preservation of Evaluation}
  %     If $M \redn N$ then $\girard{M} \redbn \girard{N}$.
  %   \end{block}
  %   \small
  %   \begin{block}{Agda Proof}
  %     \begin{code}[hide]%
\>[0]\AgdaKeyword{module}\AgdaSpace{}%
\AgdaModule{nameProofs}\AgdaSpace{}%
\AgdaKeyword{where}\<%
\\
%
\\[\AgdaEmptyExtraSkip]%
\>[0]\AgdaKeyword{open}\AgdaSpace{}%
\AgdaKeyword{import}\AgdaSpace{}%
\AgdaModule{boxCalc}\<%
\\
\>[0]\AgdaKeyword{open}\AgdaSpace{}%
\AgdaKeyword{import}\AgdaSpace{}%
\AgdaModule{calc}\<%
\\
\>[0]\AgdaKeyword{open}\AgdaSpace{}%
\AgdaKeyword{import}\AgdaSpace{}%
\AgdaModule{embedCBNIntoCBB}\<%
\\
\>[0]\AgdaKeyword{open}\AgdaSpace{}%
\AgdaKeyword{import}\AgdaSpace{}%
\AgdaModule{nameCalcRed}\<%
\\
\>[0]\AgdaKeyword{open}\AgdaSpace{}%
\AgdaKeyword{import}\AgdaSpace{}%
\AgdaModule{boxCalcRed}\<%
\\
%
\\[\AgdaEmptyExtraSkip]%
\>[0]\AgdaKeyword{open}\AgdaSpace{}%
\AgdaKeyword{import}\AgdaSpace{}%
\AgdaModule{Data.Empty}\AgdaSpace{}%
\AgdaKeyword{using}\AgdaSpace{}%
\AgdaSymbol{(}\AgdaFunction{⊥}\AgdaSymbol{;}\AgdaSpace{}%
\AgdaFunction{⊥-elim}\AgdaSymbol{)}\<%
\\
\>[0]\AgdaKeyword{import}\AgdaSpace{}%
\AgdaModule{Relation.Binary.PropositionalEquality}\AgdaSpace{}%
\AgdaSymbol{as}\AgdaSpace{}%
\AgdaModule{Eq}\<%
\\
\>[0]\AgdaKeyword{open}\AgdaSpace{}%
\AgdaModule{Eq}\AgdaSpace{}%
\AgdaKeyword{using}\AgdaSpace{}%
\AgdaSymbol{(}\AgdaOperator{\AgdaDatatype{\AgdaUnderscore{}≡\AgdaUnderscore{}}}\AgdaSymbol{;}\AgdaSpace{}%
\AgdaInductiveConstructor{refl}\AgdaSymbol{;}\AgdaSpace{}%
\AgdaFunction{cong}\AgdaSymbol{;}\AgdaSpace{}%
\AgdaFunction{cong₂}\AgdaSymbol{;}\AgdaSpace{}%
\AgdaFunction{sym}\AgdaSymbol{)}\<%
\\
%
\\[\AgdaEmptyExtraSkip]%
\>[0]\AgdaKeyword{open}\AgdaSpace{}%
\AgdaModule{Eq.≡-Reasoning}\AgdaSpace{}%
\AgdaKeyword{using}\AgdaSpace{}%
\AgdaSymbol{(}\AgdaOperator{\AgdaFunction{begin\AgdaUnderscore{}}}\AgdaSymbol{;}\AgdaSpace{}%
\AgdaFunction{step-≡-∣}\AgdaSymbol{;}\AgdaSpace{}%
\AgdaFunction{step-≡-⟩}\AgdaSymbol{;}\AgdaSpace{}%
\AgdaOperator{\AgdaFunction{\AgdaUnderscore{}∎}}\AgdaSymbol{)}\<%
\\
\>[0]\AgdaKeyword{open}\AgdaSpace{}%
\AgdaModule{Eq.≡-Reasoning}\AgdaSpace{}%
\AgdaKeyword{renaming}\AgdaSpace{}%
\AgdaSymbol{(}\AgdaOperator{\AgdaFunction{begin\AgdaUnderscore{}}}\AgdaSpace{}%
\AgdaSymbol{to}\AgdaSpace{}%
\AgdaOperator{\AgdaFunction{beginEq\AgdaUnderscore{}}}\AgdaSymbol{;}\AgdaSpace{}%
\AgdaOperator{\AgdaFunction{\AgdaUnderscore{}∎}}\AgdaSpace{}%
\AgdaSymbol{to}\AgdaSpace{}%
\AgdaOperator{\AgdaFunction{\AgdaUnderscore{}∎eq}}\AgdaSymbol{)}\<%
\\
\>[0]\AgdaKeyword{open}\AgdaSpace{}%
\AgdaKeyword{import}\AgdaSpace{}%
\AgdaModule{Function.Bundles}\AgdaSpace{}%
\AgdaKeyword{using}\AgdaSpace{}%
\AgdaSymbol{(}\AgdaOperator{\AgdaFunction{\AgdaUnderscore{}⇔\AgdaUnderscore{}}}\AgdaSymbol{;}\AgdaSpace{}%
\AgdaFunction{mk⇔}\AgdaSymbol{)}\<%
\\
%
\\[\AgdaEmptyExtraSkip]%
\>[0]\AgdaComment{--\ GIRARD'S\ embedding}\<%
\\
%
\\[\AgdaEmptyExtraSkip]%
\>[0]\AgdaComment{--\ Preservation\ of\ typing\ by\ Girard's\ translation}\<%
\\
\>[0]\AgdaFunction{pres-typing}\AgdaSpace{}%
\AgdaSymbol{:}\AgdaSpace{}%
\AgdaSymbol{∀}\AgdaSpace{}%
\AgdaSymbol{\{}\AgdaBound{Γ}\AgdaSpace{}%
\AgdaBound{A}\AgdaSymbol{\}}\AgdaSpace{}%
\AgdaSymbol{→}\AgdaSpace{}%
\AgdaBound{Γ}\AgdaSpace{}%
\AgdaOperator{\AgdaDatatype{⊢}}\AgdaSpace{}%
\AgdaBound{A}\AgdaSpace{}%
\AgdaSymbol{→}\AgdaSpace{}%
\AgdaFunction{embedContext}\AgdaSpace{}%
\AgdaBound{Γ}\AgdaSpace{}%
\AgdaOperator{\AgdaDatatype{⊢b}}\AgdaSpace{}%
\AgdaFunction{embedType}\AgdaSpace{}%
\AgdaBound{A}\<%
\\
\>[0]\AgdaFunction{pres-typing}\AgdaSpace{}%
\AgdaBound{x}\AgdaSpace{}%
\AgdaSymbol{=}\AgdaSpace{}%
\AgdaFunction{embedTerm}\AgdaSpace{}%
\AgdaBound{x}\<%
\\
%
\\[\AgdaEmptyExtraSkip]%
\>[0]\AgdaFunction{ext-embed-lookup}\AgdaSpace{}%
\AgdaSymbol{:}\AgdaSpace{}%
\AgdaSymbol{∀}\AgdaSpace{}%
\AgdaSymbol{\{}\AgdaBound{Γ}\AgdaSpace{}%
\AgdaBound{Δ}\AgdaSpace{}%
\AgdaBound{A'}\AgdaSpace{}%
\AgdaBound{A}\AgdaSymbol{\}}\<%
\\
\>[0][@{}l@{\AgdaIndent{0}}]%
\>[2]\AgdaSymbol{→}\AgdaSpace{}%
\AgdaSymbol{(}\AgdaBound{f'}\AgdaSpace{}%
\AgdaSymbol{:}\AgdaSpace{}%
\AgdaSymbol{∀}\AgdaSpace{}%
\AgdaSymbol{\{}\AgdaBound{A}\AgdaSymbol{\}}\AgdaSpace{}%
\AgdaSymbol{→}\AgdaSpace{}%
\AgdaBound{A}\AgdaSpace{}%
\AgdaOperator{\AgdaDatatype{∈}}\AgdaSpace{}%
\AgdaBound{Γ}\AgdaSpace{}%
\AgdaSymbol{→}\AgdaSpace{}%
\AgdaBound{A}\AgdaSpace{}%
\AgdaOperator{\AgdaDatatype{∈}}\AgdaSpace{}%
\AgdaBound{Δ}\AgdaSymbol{)}\<%
\\
%
\>[2]\AgdaSymbol{→}\AgdaSpace{}%
\AgdaSymbol{(}\AgdaBound{fb'}\AgdaSpace{}%
\AgdaSymbol{:}\AgdaSpace{}%
\AgdaSymbol{∀}\AgdaSpace{}%
\AgdaSymbol{\{}\AgdaBound{Ab}\AgdaSymbol{\}}\AgdaSpace{}%
\AgdaSymbol{→}\AgdaSpace{}%
\AgdaBound{Ab}\AgdaSpace{}%
\AgdaOperator{\AgdaDatatype{∈b}}\AgdaSpace{}%
\AgdaFunction{embedContext}\AgdaSpace{}%
\AgdaBound{Γ}\AgdaSpace{}%
\AgdaSymbol{→}\AgdaSpace{}%
\AgdaBound{Ab}\AgdaSpace{}%
\AgdaOperator{\AgdaDatatype{∈b}}\AgdaSpace{}%
\AgdaFunction{embedContext}\AgdaSpace{}%
\AgdaBound{Δ}\AgdaSymbol{)}\<%
\\
%
\>[2]\AgdaSymbol{→}\AgdaSpace{}%
\AgdaSymbol{(\{}\AgdaBound{A'}\AgdaSpace{}%
\AgdaSymbol{:}\AgdaSpace{}%
\AgdaDatatype{Type}\AgdaSymbol{\}}\AgdaSpace{}%
\AgdaSymbol{(}\AgdaBound{x}\AgdaSpace{}%
\AgdaSymbol{:}\AgdaSpace{}%
\AgdaBound{A'}\AgdaSpace{}%
\AgdaOperator{\AgdaDatatype{∈}}\AgdaSpace{}%
\AgdaBound{Γ}\AgdaSymbol{)}\AgdaSpace{}%
\AgdaSymbol{→}\AgdaSpace{}%
\AgdaBound{fb'}\AgdaSpace{}%
\AgdaSymbol{(}\AgdaFunction{embedLookup}\AgdaSpace{}%
\AgdaBound{x}\AgdaSymbol{)}\AgdaSpace{}%
\AgdaOperator{\AgdaDatatype{≡}}\AgdaSpace{}%
\AgdaFunction{embedLookup}\AgdaSpace{}%
\AgdaSymbol{(}\AgdaBound{f'}\AgdaSpace{}%
\AgdaBound{x}\AgdaSymbol{))}\<%
\\
%
\>[2]\AgdaSymbol{→}\AgdaSpace{}%
\AgdaSymbol{(}\AgdaBound{x}\AgdaSpace{}%
\AgdaSymbol{:}\AgdaSpace{}%
\AgdaBound{A'}\AgdaSpace{}%
\AgdaOperator{\AgdaDatatype{∈}}\AgdaSpace{}%
\AgdaBound{Γ}\AgdaSpace{}%
\AgdaOperator{\AgdaInductiveConstructor{,}}\AgdaSpace{}%
\AgdaBound{A}\AgdaSymbol{)}\<%
\\
%
\>[2]\AgdaSymbol{→}\AgdaSpace{}%
\AgdaFunction{extb}\AgdaSpace{}%
\AgdaBound{fb'}\AgdaSpace{}%
\AgdaSymbol{(}\AgdaFunction{embedLookup}\AgdaSpace{}%
\AgdaBound{x}\AgdaSymbol{)}\AgdaSpace{}%
\AgdaOperator{\AgdaDatatype{≡}}\AgdaSpace{}%
\AgdaFunction{embedLookup}\AgdaSpace{}%
\AgdaSymbol{(}\AgdaFunction{ext}\AgdaSpace{}%
\AgdaBound{f'}\AgdaSpace{}%
\AgdaBound{x}\AgdaSymbol{)}\<%
\\
\>[0]\AgdaFunction{ext-embed-lookup}\AgdaSpace{}%
\AgdaBound{f'}\AgdaSpace{}%
\AgdaBound{fb'}\AgdaSpace{}%
\AgdaBound{h}\AgdaSpace{}%
\AgdaInductiveConstructor{Z}\AgdaSpace{}%
\AgdaSymbol{=}\AgdaSpace{}%
\AgdaInductiveConstructor{refl}\<%
\\
\>[0]\AgdaFunction{ext-embed-lookup}\AgdaSpace{}%
\AgdaBound{f'}\AgdaSpace{}%
\AgdaBound{fb'}\AgdaSpace{}%
\AgdaBound{h}\AgdaSpace{}%
\AgdaSymbol{(}\AgdaInductiveConstructor{S}\AgdaSpace{}%
\AgdaBound{x}\AgdaSymbol{)}\AgdaSpace{}%
\AgdaSymbol{=}\AgdaSpace{}%
\AgdaFunction{cong}\AgdaSpace{}%
\AgdaInductiveConstructor{Sb}\AgdaSpace{}%
\AgdaSymbol{(}\AgdaBound{h}\AgdaSpace{}%
\AgdaBound{x}\AgdaSymbol{)}\<%
\\
%
\\[\AgdaEmptyExtraSkip]%
\>[0]\AgdaFunction{pres-rename}\AgdaSpace{}%
\AgdaSymbol{:}\AgdaSpace{}%
\AgdaSymbol{∀}\AgdaSpace{}%
\AgdaSymbol{\{}\AgdaBound{Γ}\AgdaSpace{}%
\AgdaBound{A}\AgdaSpace{}%
\AgdaBound{Δ}\AgdaSymbol{\}}\<%
\\
\>[0][@{}l@{\AgdaIndent{0}}]%
\>[2]\AgdaSymbol{→}\AgdaSpace{}%
\AgdaSymbol{(}\AgdaBound{f'}\AgdaSpace{}%
\AgdaSymbol{:}\AgdaSpace{}%
\AgdaSymbol{∀}\AgdaSpace{}%
\AgdaSymbol{\{}\AgdaBound{A}\AgdaSymbol{\}}\AgdaSpace{}%
\AgdaSymbol{→}\AgdaSpace{}%
\AgdaBound{A}\AgdaSpace{}%
\AgdaOperator{\AgdaDatatype{∈}}\AgdaSpace{}%
\AgdaBound{Γ}\AgdaSpace{}%
\AgdaSymbol{→}\AgdaSpace{}%
\AgdaBound{A}\AgdaSpace{}%
\AgdaOperator{\AgdaDatatype{∈}}\AgdaSpace{}%
\AgdaBound{Δ}\AgdaSymbol{)}\<%
\\
%
\>[2]\AgdaSymbol{→}\AgdaSpace{}%
\AgdaSymbol{(}\AgdaBound{fb'}\AgdaSpace{}%
\AgdaSymbol{:}\AgdaSpace{}%
\AgdaSymbol{∀}\AgdaSpace{}%
\AgdaSymbol{\{}\AgdaBound{Ab}\AgdaSymbol{\}}\AgdaSpace{}%
\AgdaSymbol{→}\AgdaSpace{}%
\AgdaBound{Ab}\AgdaSpace{}%
\AgdaOperator{\AgdaDatatype{∈b}}\AgdaSpace{}%
\AgdaFunction{embedContext}\AgdaSpace{}%
\AgdaBound{Γ}\AgdaSpace{}%
\AgdaSymbol{→}\AgdaSpace{}%
\AgdaBound{Ab}\AgdaSpace{}%
\AgdaOperator{\AgdaDatatype{∈b}}\AgdaSpace{}%
\AgdaFunction{embedContext}\AgdaSpace{}%
\AgdaBound{Δ}\AgdaSymbol{)}\<%
\\
%
\>[2]\AgdaSymbol{→}\AgdaSpace{}%
\AgdaSymbol{(\{}\AgdaBound{A'}\AgdaSpace{}%
\AgdaSymbol{:}\AgdaSpace{}%
\AgdaDatatype{Type}\AgdaSymbol{\}}\AgdaSpace{}%
\AgdaSymbol{(}\AgdaBound{x}\AgdaSpace{}%
\AgdaSymbol{:}\AgdaSpace{}%
\AgdaBound{A'}\AgdaSpace{}%
\AgdaOperator{\AgdaDatatype{∈}}\AgdaSpace{}%
\AgdaBound{Γ}\AgdaSymbol{)}\AgdaSpace{}%
\AgdaSymbol{→}\AgdaSpace{}%
\AgdaBound{fb'}\AgdaSpace{}%
\AgdaSymbol{(}\AgdaFunction{embedLookup}\AgdaSpace{}%
\AgdaBound{x}\AgdaSymbol{)}\AgdaSpace{}%
\AgdaOperator{\AgdaDatatype{≡}}\AgdaSpace{}%
\AgdaFunction{embedLookup}\AgdaSpace{}%
\AgdaSymbol{(}\AgdaBound{f'}\AgdaSpace{}%
\AgdaBound{x}\AgdaSymbol{))}\<%
\\
%
\>[2]\AgdaSymbol{→}\AgdaSpace{}%
\AgdaSymbol{(}\AgdaBound{t}\AgdaSpace{}%
\AgdaSymbol{:}\AgdaSpace{}%
\AgdaBound{Γ}\AgdaSpace{}%
\AgdaOperator{\AgdaDatatype{⊢}}\AgdaSpace{}%
\AgdaBound{A}\AgdaSymbol{)}\<%
\\
%
\>[2]\AgdaSymbol{→}\AgdaSpace{}%
\AgdaFunction{boxCalc.rename}\AgdaSpace{}%
\AgdaBound{fb'}\AgdaSpace{}%
\AgdaSymbol{(}\AgdaFunction{embedTerm}\AgdaSpace{}%
\AgdaBound{t}\AgdaSymbol{)}\AgdaSpace{}%
\AgdaOperator{\AgdaDatatype{≡}}\AgdaSpace{}%
\AgdaFunction{embedTerm}\AgdaSpace{}%
\AgdaSymbol{(}\AgdaFunction{calc.rename}\AgdaSpace{}%
\AgdaBound{f'}\AgdaSpace{}%
\AgdaBound{t}\AgdaSymbol{)}\<%
\\
\>[0]\AgdaFunction{pres-rename}\AgdaSpace{}%
\AgdaBound{f'}\AgdaSpace{}%
\AgdaBound{fb'}\AgdaSpace{}%
\AgdaBound{h}\AgdaSpace{}%
\AgdaSymbol{(}\AgdaInductiveConstructor{`}\AgdaSpace{}%
\AgdaBound{x}\AgdaSymbol{)}\AgdaSpace{}%
\AgdaSymbol{=}\<%
\\
\>[0][@{}l@{\AgdaIndent{0}}]%
\>[2]\AgdaFunction{cong}\AgdaSpace{}%
\AgdaInductiveConstructor{ε}\AgdaSpace{}%
\AgdaSymbol{(}\AgdaBound{h}\AgdaSpace{}%
\AgdaBound{x}\AgdaSymbol{)}\<%
\\
\>[0]\AgdaFunction{pres-rename}\AgdaSpace{}%
\AgdaBound{f'}\AgdaSpace{}%
\AgdaBound{fb'}\AgdaSpace{}%
\AgdaBound{h}\AgdaSpace{}%
\AgdaSymbol{(}\AgdaInductiveConstructor{ƛ}\AgdaSpace{}%
\AgdaBound{t}\AgdaSymbol{)}\AgdaSpace{}%
\AgdaSymbol{=}\<%
\\
\>[0][@{}l@{\AgdaIndent{0}}]%
\>[2]\AgdaFunction{cong}\AgdaSpace{}%
\AgdaInductiveConstructor{ƛb}\AgdaSpace{}%
\AgdaSymbol{(}\AgdaFunction{pres-rename}\AgdaSpace{}%
\AgdaSymbol{(}\AgdaFunction{calc.ext}\AgdaSpace{}%
\AgdaBound{f'}\AgdaSymbol{)}\AgdaSpace{}%
\AgdaSymbol{(}\AgdaFunction{boxCalc.extb}\AgdaSpace{}%
\AgdaBound{fb'}\AgdaSymbol{)}\AgdaSpace{}%
\AgdaSymbol{(}\AgdaFunction{ext-embed-lookup}\AgdaSpace{}%
\AgdaBound{f'}\AgdaSpace{}%
\AgdaBound{fb'}\AgdaSpace{}%
\AgdaBound{h}\AgdaSymbol{)}\AgdaSpace{}%
\AgdaBound{t}\AgdaSymbol{)}\<%
\\
\>[0]\AgdaFunction{pres-rename}\AgdaSpace{}%
\AgdaBound{f'}\AgdaSpace{}%
\AgdaBound{fb'}\AgdaSpace{}%
\AgdaBound{h}\AgdaSpace{}%
\AgdaSymbol{(}\AgdaBound{t}\AgdaSpace{}%
\AgdaOperator{\AgdaInductiveConstructor{·}}\AgdaSpace{}%
\AgdaBound{t₁}\AgdaSymbol{)}\AgdaSpace{}%
\AgdaSymbol{=}\<%
\\
\>[0][@{}l@{\AgdaIndent{0}}]%
\>[2]\AgdaFunction{cong₂}\AgdaSpace{}%
\AgdaOperator{\AgdaInductiveConstructor{\AgdaUnderscore{}·b\AgdaUnderscore{}}}\AgdaSpace{}%
\AgdaSymbol{(}\AgdaFunction{pres-rename}\AgdaSpace{}%
\AgdaBound{f'}\AgdaSpace{}%
\AgdaBound{fb'}\AgdaSpace{}%
\AgdaBound{h}\AgdaSpace{}%
\AgdaBound{t}\AgdaSymbol{)}\AgdaSpace{}%
\AgdaSymbol{(}\AgdaFunction{cong}\AgdaSpace{}%
\AgdaInductiveConstructor{box}\AgdaSpace{}%
\AgdaSymbol{(}\AgdaFunction{pres-rename}\AgdaSpace{}%
\AgdaBound{f'}\AgdaSpace{}%
\AgdaBound{fb'}\AgdaSpace{}%
\AgdaBound{h}\AgdaSpace{}%
\AgdaBound{t₁}\AgdaSymbol{))}\<%
\\
%
\\[\AgdaEmptyExtraSkip]%
%
\\[\AgdaEmptyExtraSkip]%
\>[0]\AgdaFunction{exts-pres-rename}\AgdaSpace{}%
\AgdaSymbol{:}\AgdaSpace{}%
\AgdaSymbol{∀}\AgdaSpace{}%
\AgdaSymbol{\{}\AgdaBound{Γ}\AgdaSpace{}%
\AgdaBound{Δ}\AgdaSymbol{\}}\<%
\\
\>[0][@{}l@{\AgdaIndent{0}}]%
\>[2]\AgdaSymbol{→}\AgdaSpace{}%
\AgdaSymbol{(}\AgdaBound{fb'}\AgdaSpace{}%
\AgdaSymbol{:}\AgdaSpace{}%
\AgdaSymbol{∀}\AgdaSpace{}%
\AgdaSymbol{\{}\AgdaBound{Ab}\AgdaSymbol{\}}\AgdaSpace{}%
\AgdaSymbol{→}\AgdaSpace{}%
\AgdaInductiveConstructor{□}\AgdaSpace{}%
\AgdaBound{Ab}\AgdaSpace{}%
\AgdaOperator{\AgdaDatatype{∈b}}\AgdaSpace{}%
\AgdaFunction{embedContext}\AgdaSpace{}%
\AgdaBound{Γ}\AgdaSpace{}%
\AgdaSymbol{→}\AgdaSpace{}%
\AgdaFunction{embedContext}\AgdaSpace{}%
\AgdaBound{Δ}\AgdaSpace{}%
\AgdaOperator{\AgdaDatatype{⊢b}}\AgdaSpace{}%
\AgdaBound{Ab}\AgdaSymbol{)}\<%
\\
%
\>[2]\AgdaSymbol{→}\AgdaSpace{}%
\AgdaSymbol{(}\AgdaBound{f'}\AgdaSpace{}%
\AgdaSymbol{:}\AgdaSpace{}%
\AgdaSymbol{∀}\AgdaSpace{}%
\AgdaSymbol{\{}\AgdaBound{A}\AgdaSymbol{\}}\AgdaSpace{}%
\AgdaSymbol{→}\AgdaSpace{}%
\AgdaBound{A}\AgdaSpace{}%
\AgdaOperator{\AgdaDatatype{∈}}\AgdaSpace{}%
\AgdaBound{Γ}\AgdaSpace{}%
\AgdaSymbol{→}\AgdaSpace{}%
\AgdaBound{Δ}\AgdaSpace{}%
\AgdaOperator{\AgdaDatatype{⊢}}\AgdaSpace{}%
\AgdaBound{A}\AgdaSymbol{)}\<%
\\
%
\>[2]\AgdaSymbol{→}\AgdaSpace{}%
\AgdaSymbol{(\{}\AgdaBound{A'}\AgdaSpace{}%
\AgdaSymbol{:}\AgdaSpace{}%
\AgdaDatatype{Type}\AgdaSymbol{\}}\AgdaSpace{}%
\AgdaSymbol{(}\AgdaBound{x}\AgdaSpace{}%
\AgdaSymbol{:}\AgdaSpace{}%
\AgdaBound{A'}\AgdaSpace{}%
\AgdaOperator{\AgdaDatatype{∈}}\AgdaSpace{}%
\AgdaBound{Γ}\AgdaSymbol{)}\AgdaSpace{}%
\AgdaSymbol{→}\AgdaSpace{}%
\AgdaBound{fb'}\AgdaSpace{}%
\AgdaSymbol{(}\AgdaFunction{embedLookup}\AgdaSpace{}%
\AgdaBound{x}\AgdaSymbol{)}\AgdaSpace{}%
\AgdaOperator{\AgdaDatatype{≡}}\AgdaSpace{}%
\AgdaFunction{embedTerm}\AgdaSpace{}%
\AgdaSymbol{(}\AgdaBound{f'}\AgdaSpace{}%
\AgdaBound{x}\AgdaSymbol{))}\<%
\\
%
\>[2]\AgdaSymbol{→}\AgdaSpace{}%
\AgdaSymbol{\{}\AgdaBound{A'}\AgdaSpace{}%
\AgdaBound{A}\AgdaSpace{}%
\AgdaSymbol{:}\AgdaSpace{}%
\AgdaDatatype{Type}\AgdaSymbol{\}}\AgdaSpace{}%
\AgdaSymbol{(}\AgdaBound{x}\AgdaSpace{}%
\AgdaSymbol{:}\AgdaSpace{}%
\AgdaBound{A'}\AgdaSpace{}%
\AgdaOperator{\AgdaDatatype{∈}}\AgdaSpace{}%
\AgdaBound{Γ}\AgdaSpace{}%
\AgdaOperator{\AgdaInductiveConstructor{,}}\AgdaSpace{}%
\AgdaBound{A}\AgdaSymbol{)}\AgdaSpace{}%
\AgdaSymbol{→}\AgdaSpace{}%
\AgdaFunction{boxCalc.exts}\AgdaSpace{}%
\AgdaBound{fb'}\AgdaSpace{}%
\AgdaSymbol{(}\AgdaFunction{embedLookup}\AgdaSpace{}%
\AgdaBound{x}\AgdaSymbol{)}\AgdaSpace{}%
\AgdaOperator{\AgdaDatatype{≡}}\AgdaSpace{}%
\AgdaFunction{embedTerm}\AgdaSpace{}%
\AgdaSymbol{(}\AgdaFunction{calc.exts}\AgdaSpace{}%
\AgdaBound{f'}\AgdaSpace{}%
\AgdaBound{x}\AgdaSymbol{)}\<%
\\
\>[0]\AgdaFunction{exts-pres-rename}\AgdaSpace{}%
\AgdaBound{fb'}\AgdaSpace{}%
\AgdaBound{f'}\AgdaSpace{}%
\AgdaBound{h}\AgdaSpace{}%
\AgdaInductiveConstructor{Z}\AgdaSpace{}%
\AgdaSymbol{=}\AgdaSpace{}%
\AgdaInductiveConstructor{refl}\<%
\\
\>[0]\AgdaFunction{exts-pres-rename}\AgdaSpace{}%
\AgdaBound{fb'}\AgdaSpace{}%
\AgdaBound{f'}\AgdaSpace{}%
\AgdaBound{h}\AgdaSpace{}%
\AgdaSymbol{(}\AgdaInductiveConstructor{S}\AgdaSpace{}%
\AgdaBound{x}\AgdaSymbol{)}\AgdaSpace{}%
\AgdaSymbol{=}\<%
\\
\>[0][@{}l@{\AgdaIndent{0}}]%
\>[2]\AgdaOperator{\AgdaFunction{beginEq}}\<%
\\
\>[2][@{}l@{\AgdaIndent{0}}]%
\>[4]\AgdaFunction{boxCalc.rename}\AgdaSpace{}%
\AgdaInductiveConstructor{Sb}\AgdaSpace{}%
\AgdaSymbol{(}\AgdaBound{fb'}\AgdaSpace{}%
\AgdaSymbol{(}\AgdaFunction{embedLookup}\AgdaSpace{}%
\AgdaBound{x}\AgdaSymbol{))}\<%
\\
%
\>[2]\AgdaFunction{≡⟨}\AgdaSpace{}%
\AgdaFunction{cong}\AgdaSpace{}%
\AgdaSymbol{(}\AgdaFunction{boxCalc.rename}\AgdaSpace{}%
\AgdaInductiveConstructor{Sb}\AgdaSymbol{)}\AgdaSpace{}%
\AgdaSymbol{(}\AgdaBound{h}\AgdaSpace{}%
\AgdaBound{x}\AgdaSymbol{)}\AgdaSpace{}%
\AgdaFunction{⟩}\<%
\\
\>[2][@{}l@{\AgdaIndent{0}}]%
\>[4]\AgdaFunction{boxCalc.rename}\AgdaSpace{}%
\AgdaInductiveConstructor{Sb}\AgdaSpace{}%
\AgdaSymbol{(}\AgdaFunction{embedTerm}\AgdaSpace{}%
\AgdaSymbol{(}\AgdaBound{f'}\AgdaSpace{}%
\AgdaBound{x}\AgdaSymbol{))}\<%
\\
%
\>[2]\AgdaFunction{≡⟨}\AgdaSpace{}%
\AgdaFunction{pres-rename}\AgdaSpace{}%
\AgdaInductiveConstructor{S}\AgdaSpace{}%
\AgdaInductiveConstructor{Sb}\AgdaSpace{}%
\AgdaSymbol{(λ}\AgdaSpace{}%
\AgdaBound{x₁}\AgdaSpace{}%
\AgdaSymbol{→}\AgdaSpace{}%
\AgdaInductiveConstructor{refl}\AgdaSymbol{)}\AgdaSpace{}%
\AgdaSymbol{(}\AgdaBound{f'}\AgdaSpace{}%
\AgdaBound{x}\AgdaSymbol{)}\AgdaSpace{}%
\AgdaFunction{⟩}\<%
\\
\>[2][@{}l@{\AgdaIndent{0}}]%
\>[4]\AgdaFunction{embedTerm}\AgdaSpace{}%
\AgdaSymbol{(}\AgdaFunction{calc.rename}\AgdaSpace{}%
\AgdaInductiveConstructor{S}\AgdaSpace{}%
\AgdaSymbol{(}\AgdaBound{f'}\AgdaSpace{}%
\AgdaBound{x}\AgdaSymbol{))}\<%
\\
%
\>[2]\AgdaOperator{\AgdaFunction{∎eq}}\<%
\\
%
\\[\AgdaEmptyExtraSkip]%
\>[0]\AgdaFunction{pres-simultaneous-subst}\AgdaSpace{}%
\AgdaSymbol{:}\AgdaSpace{}%
\AgdaSymbol{∀}\AgdaSpace{}%
\AgdaSymbol{\{}\AgdaBound{Γ}\AgdaSpace{}%
\AgdaBound{Δ}\AgdaSpace{}%
\AgdaBound{A}\AgdaSymbol{\}}\<%
\\
\>[0][@{}l@{\AgdaIndent{0}}]%
\>[2]\AgdaSymbol{→}\AgdaSpace{}%
\AgdaSymbol{(}\AgdaBound{fb'}\AgdaSpace{}%
\AgdaSymbol{:}\AgdaSpace{}%
\AgdaSymbol{∀}\AgdaSpace{}%
\AgdaSymbol{\{}\AgdaBound{Ab}\AgdaSymbol{\}}\AgdaSpace{}%
\AgdaSymbol{→}\AgdaSpace{}%
\AgdaInductiveConstructor{□}\AgdaSpace{}%
\AgdaBound{Ab}\AgdaSpace{}%
\AgdaOperator{\AgdaDatatype{∈b}}\AgdaSpace{}%
\AgdaFunction{embedContext}\AgdaSpace{}%
\AgdaBound{Γ}\AgdaSpace{}%
\AgdaSymbol{→}\AgdaSpace{}%
\AgdaFunction{embedContext}\AgdaSpace{}%
\AgdaBound{Δ}\AgdaSpace{}%
\AgdaOperator{\AgdaDatatype{⊢b}}\AgdaSpace{}%
\AgdaBound{Ab}\AgdaSymbol{)}\<%
\\
%
\>[2]\AgdaSymbol{→}\AgdaSpace{}%
\AgdaSymbol{(}\AgdaBound{f'}\AgdaSpace{}%
\AgdaSymbol{:}\AgdaSpace{}%
\AgdaSymbol{∀}\AgdaSpace{}%
\AgdaSymbol{\{}\AgdaBound{A}\AgdaSymbol{\}}\AgdaSpace{}%
\AgdaSymbol{→}\AgdaSpace{}%
\AgdaBound{A}\AgdaSpace{}%
\AgdaOperator{\AgdaDatatype{∈}}\AgdaSpace{}%
\AgdaBound{Γ}\AgdaSpace{}%
\AgdaSymbol{→}\AgdaSpace{}%
\AgdaBound{Δ}\AgdaSpace{}%
\AgdaOperator{\AgdaDatatype{⊢}}\AgdaSpace{}%
\AgdaBound{A}\AgdaSymbol{)}\<%
\\
%
\>[2]\AgdaSymbol{→}\AgdaSpace{}%
\AgdaSymbol{(}\AgdaBound{h}\AgdaSpace{}%
\AgdaSymbol{:}\AgdaSpace{}%
\AgdaSymbol{∀}\AgdaSpace{}%
\AgdaSymbol{\{}\AgdaBound{A'}\AgdaSymbol{\}}\AgdaSpace{}%
\AgdaSymbol{→}\AgdaSpace{}%
\AgdaSymbol{(}\AgdaBound{x}\AgdaSpace{}%
\AgdaSymbol{:}\AgdaSpace{}%
\AgdaBound{A'}\AgdaSpace{}%
\AgdaOperator{\AgdaDatatype{∈}}\AgdaSpace{}%
\AgdaBound{Γ}\AgdaSymbol{)}\AgdaSpace{}%
\AgdaSymbol{→}\AgdaSpace{}%
\AgdaBound{fb'}\AgdaSpace{}%
\AgdaSymbol{(}\AgdaFunction{embedLookup}\AgdaSpace{}%
\AgdaBound{x}\AgdaSymbol{)}\AgdaSpace{}%
\AgdaOperator{\AgdaDatatype{≡}}\AgdaSpace{}%
\AgdaFunction{embedTerm}\AgdaSpace{}%
\AgdaSymbol{(}\AgdaBound{f'}\AgdaSpace{}%
\AgdaBound{x}\AgdaSymbol{)}\AgdaSpace{}%
\AgdaSymbol{)}\<%
\\
%
\>[2]\AgdaSymbol{→}\AgdaSpace{}%
\AgdaSymbol{(}\AgdaBound{M}\AgdaSpace{}%
\AgdaSymbol{:}\AgdaSpace{}%
\AgdaBound{Γ}\AgdaSpace{}%
\AgdaOperator{\AgdaDatatype{⊢}}\AgdaSpace{}%
\AgdaBound{A}\AgdaSymbol{)}\<%
\\
%
\>[2]\AgdaSymbol{→}\AgdaSpace{}%
\AgdaFunction{substb}\AgdaSpace{}%
\AgdaBound{fb'}\AgdaSpace{}%
\AgdaSymbol{(}\AgdaFunction{embedTerm}\AgdaSpace{}%
\AgdaBound{M}\AgdaSymbol{)}\AgdaSpace{}%
\AgdaOperator{\AgdaDatatype{≡}}\AgdaSpace{}%
\AgdaFunction{embedTerm}\AgdaSpace{}%
\AgdaSymbol{(}\AgdaFunction{subst}\AgdaSpace{}%
\AgdaBound{f'}\AgdaSpace{}%
\AgdaBound{M}\AgdaSymbol{)}\<%
\\
\>[0]\AgdaFunction{pres-simultaneous-subst}\AgdaSpace{}%
\AgdaBound{fb'}\AgdaSpace{}%
\AgdaBound{f'}\AgdaSpace{}%
\AgdaBound{h}\AgdaSpace{}%
\AgdaSymbol{(}\AgdaInductiveConstructor{`}\AgdaSpace{}%
\AgdaBound{x}\AgdaSymbol{)}\AgdaSpace{}%
\AgdaSymbol{=}\<%
\\
\>[0][@{}l@{\AgdaIndent{0}}]%
\>[2]\AgdaBound{h}\AgdaSpace{}%
\AgdaBound{x}\<%
\\
\>[0]\AgdaFunction{pres-simultaneous-subst}\AgdaSpace{}%
\AgdaBound{fb'}\AgdaSpace{}%
\AgdaBound{f'}\AgdaSpace{}%
\AgdaBound{h}\AgdaSpace{}%
\AgdaSymbol{(}\AgdaInductiveConstructor{ƛ}\AgdaSpace{}%
\AgdaBound{M}\AgdaSymbol{)}\AgdaSpace{}%
\AgdaSymbol{=}\<%
\\
\>[0][@{}l@{\AgdaIndent{0}}]%
\>[2]\AgdaFunction{cong}\AgdaSpace{}%
\AgdaInductiveConstructor{ƛb}\AgdaSpace{}%
\AgdaSymbol{(}\AgdaSpace{}%
\AgdaFunction{pres-simultaneous-subst}\AgdaSpace{}%
\AgdaSymbol{(}\AgdaFunction{boxCalc.exts}\AgdaSpace{}%
\AgdaBound{fb'}\AgdaSymbol{)}\AgdaSpace{}%
\AgdaSymbol{(}\AgdaFunction{calc.exts}\AgdaSpace{}%
\AgdaBound{f'}\AgdaSymbol{)}\AgdaSpace{}%
\AgdaSymbol{(}\AgdaFunction{exts-pres-rename}\AgdaSpace{}%
\AgdaBound{fb'}\AgdaSpace{}%
\AgdaBound{f'}\AgdaSpace{}%
\AgdaBound{h}\AgdaSymbol{)}\AgdaSpace{}%
\AgdaBound{M}\AgdaSpace{}%
\AgdaSymbol{)}\<%
\\
\>[0]\AgdaFunction{pres-simultaneous-subst}\AgdaSpace{}%
\AgdaBound{fb'}\AgdaSpace{}%
\AgdaBound{f'}\AgdaSpace{}%
\AgdaBound{h}\AgdaSpace{}%
\AgdaSymbol{(}\AgdaBound{M}\AgdaSpace{}%
\AgdaOperator{\AgdaInductiveConstructor{·}}\AgdaSpace{}%
\AgdaBound{M₁}\AgdaSymbol{)}\AgdaSpace{}%
\AgdaSymbol{=}\<%
\\
\>[0][@{}l@{\AgdaIndent{0}}]%
\>[2]\AgdaFunction{cong₂}\AgdaSpace{}%
\AgdaOperator{\AgdaInductiveConstructor{\AgdaUnderscore{}·b\AgdaUnderscore{}}}\AgdaSpace{}%
\AgdaSymbol{(}\AgdaFunction{pres-simultaneous-subst}\AgdaSpace{}%
\AgdaBound{fb'}\AgdaSpace{}%
\AgdaBound{f'}\AgdaSpace{}%
\AgdaBound{h}\AgdaSpace{}%
\AgdaBound{M}\AgdaSymbol{)}\AgdaSpace{}%
\AgdaSymbol{(}\AgdaFunction{cong}\AgdaSpace{}%
\AgdaInductiveConstructor{box}\AgdaSpace{}%
\AgdaSymbol{(}\AgdaFunction{pres-simultaneous-subst}\AgdaSpace{}%
\AgdaBound{fb'}\AgdaSpace{}%
\AgdaBound{f'}\AgdaSpace{}%
\AgdaBound{h}\AgdaSpace{}%
\AgdaBound{M₁}\AgdaSymbol{))}\<%
\\
%
\\[\AgdaEmptyExtraSkip]%
%
\\[\AgdaEmptyExtraSkip]%
\>[0]\AgdaComment{--\ Proof\ of\ h}\<%
\\
\>[0]\AgdaFunction{pres-single-subst}\AgdaSpace{}%
\AgdaSymbol{:}\AgdaSpace{}%
\AgdaSymbol{∀}\AgdaSpace{}%
\AgdaSymbol{\{}\AgdaBound{A}\AgdaSpace{}%
\AgdaBound{C}\AgdaSpace{}%
\AgdaBound{Γ}\AgdaSymbol{\}}\<%
\\
\>[0][@{}l@{\AgdaIndent{0}}]%
\>[2]\AgdaSymbol{→}\AgdaSpace{}%
\AgdaSymbol{(}\AgdaBound{N}\AgdaSpace{}%
\AgdaSymbol{:}\AgdaSpace{}%
\AgdaBound{Γ}\AgdaSpace{}%
\AgdaOperator{\AgdaDatatype{⊢}}\AgdaSpace{}%
\AgdaBound{C}\AgdaSymbol{)}\<%
\\
%
\>[2]\AgdaSymbol{→}\AgdaSpace{}%
\AgdaSymbol{(}\AgdaBound{x}\AgdaSpace{}%
\AgdaSymbol{:}\AgdaSpace{}%
\AgdaBound{A}\AgdaSpace{}%
\AgdaOperator{\AgdaDatatype{∈}}\AgdaSpace{}%
\AgdaBound{Γ}\AgdaSpace{}%
\AgdaOperator{\AgdaInductiveConstructor{,}}\AgdaSpace{}%
\AgdaBound{C}\AgdaSymbol{)}\<%
\\
%
\>[2]\AgdaSymbol{→}\AgdaSpace{}%
\AgdaFunction{fb}\AgdaSpace{}%
\AgdaSymbol{(}\AgdaFunction{embedTerm}\AgdaSpace{}%
\AgdaBound{N}\AgdaSymbol{)}\AgdaSpace{}%
\AgdaSymbol{(}\AgdaFunction{embedLookup}\AgdaSpace{}%
\AgdaBound{x}\AgdaSymbol{)}\AgdaSpace{}%
\AgdaOperator{\AgdaDatatype{≡}}\AgdaSpace{}%
\AgdaFunction{embedTerm}\AgdaSpace{}%
\AgdaSymbol{((}\AgdaFunction{f}\AgdaSpace{}%
\AgdaBound{N}\AgdaSymbol{)}\AgdaSpace{}%
\AgdaBound{x}\AgdaSymbol{)}\<%
\\
\>[0]\AgdaFunction{pres-single-subst}\AgdaSpace{}%
\AgdaBound{N}\AgdaSpace{}%
\AgdaInductiveConstructor{Z}\AgdaSpace{}%
\AgdaSymbol{=}\AgdaSpace{}%
\AgdaInductiveConstructor{refl}\<%
\\
\>[0]\AgdaFunction{pres-single-subst}\AgdaSpace{}%
\AgdaBound{N}\AgdaSpace{}%
\AgdaSymbol{(}\AgdaInductiveConstructor{S}\AgdaSpace{}%
\AgdaBound{x}\AgdaSymbol{)}\AgdaSpace{}%
\AgdaSymbol{=}\AgdaSpace{}%
\AgdaInductiveConstructor{refl}\<%
\\
%
\\[\AgdaEmptyExtraSkip]%
\>[0]\AgdaFunction{exts-pres-single-subst}\AgdaSpace{}%
\AgdaSymbol{:}\AgdaSpace{}%
\AgdaSymbol{∀}\AgdaSpace{}%
\AgdaSymbol{\{}\AgdaBound{A}\AgdaSpace{}%
\AgdaBound{C}\AgdaSpace{}%
\AgdaBound{Γ}\AgdaSpace{}%
\AgdaBound{B}\AgdaSymbol{\}}\<%
\\
\>[0][@{}l@{\AgdaIndent{0}}]%
\>[2]\AgdaSymbol{→}\AgdaSpace{}%
\AgdaSymbol{(}\AgdaBound{N}\AgdaSpace{}%
\AgdaSymbol{:}\AgdaSpace{}%
\AgdaBound{Γ}\AgdaSpace{}%
\AgdaOperator{\AgdaDatatype{⊢}}\AgdaSpace{}%
\AgdaBound{C}\AgdaSymbol{)}\<%
\\
%
\>[2]\AgdaSymbol{→}\AgdaSpace{}%
\AgdaSymbol{(}\AgdaBound{x}\AgdaSpace{}%
\AgdaSymbol{:}\AgdaSpace{}%
\AgdaBound{A}\AgdaSpace{}%
\AgdaOperator{\AgdaDatatype{∈}}\AgdaSpace{}%
\AgdaBound{Γ}\AgdaSpace{}%
\AgdaOperator{\AgdaInductiveConstructor{,}}\AgdaSpace{}%
\AgdaBound{C}\AgdaSpace{}%
\AgdaOperator{\AgdaInductiveConstructor{,}}\AgdaSpace{}%
\AgdaBound{B}\AgdaSymbol{)}\<%
\\
%
\>[2]\AgdaSymbol{→}\AgdaSpace{}%
\AgdaFunction{boxCalc.exts}\AgdaSpace{}%
\AgdaSymbol{(}\AgdaFunction{fb}\AgdaSpace{}%
\AgdaSymbol{(}\AgdaFunction{embedTerm}\AgdaSpace{}%
\AgdaBound{N}\AgdaSymbol{))}\AgdaSpace{}%
\AgdaSymbol{(}\AgdaFunction{embedLookup}\AgdaSpace{}%
\AgdaBound{x}\AgdaSymbol{)}\AgdaSpace{}%
\AgdaOperator{\AgdaDatatype{≡}}\AgdaSpace{}%
\AgdaFunction{embedTerm}\AgdaSpace{}%
\AgdaSymbol{(}\AgdaFunction{calc.exts}\AgdaSpace{}%
\AgdaSymbol{(}\AgdaFunction{f}\AgdaSpace{}%
\AgdaBound{N}\AgdaSymbol{)}\AgdaSpace{}%
\AgdaBound{x}\AgdaSymbol{)}\<%
\\
\>[0]\AgdaFunction{exts-pres-single-subst}\AgdaSpace{}%
\AgdaBound{N}\AgdaSpace{}%
\AgdaInductiveConstructor{Z}\AgdaSpace{}%
\AgdaSymbol{=}\AgdaSpace{}%
\AgdaInductiveConstructor{refl}\<%
\\
\>[0]\AgdaFunction{exts-pres-single-subst}\AgdaSpace{}%
\AgdaBound{N}\AgdaSpace{}%
\AgdaSymbol{(}\AgdaInductiveConstructor{S}\AgdaSpace{}%
\AgdaBound{x}\AgdaSymbol{)}\AgdaSpace{}%
\AgdaSymbol{=}\<%
\\
\>[0][@{}l@{\AgdaIndent{0}}]%
\>[2]\AgdaOperator{\AgdaFunction{beginEq}}\<%
\\
\>[2][@{}l@{\AgdaIndent{0}}]%
\>[4]\AgdaFunction{boxCalc.rename}\AgdaSpace{}%
\AgdaInductiveConstructor{Sb}\AgdaSpace{}%
\AgdaSymbol{(}\AgdaFunction{fb}\AgdaSpace{}%
\AgdaSymbol{(}\AgdaFunction{embedTerm}\AgdaSpace{}%
\AgdaBound{N}\AgdaSymbol{)}\AgdaSpace{}%
\AgdaSymbol{(}\AgdaFunction{embedLookup}\AgdaSpace{}%
\AgdaBound{x}\AgdaSymbol{))}\<%
\\
%
\>[2]\AgdaFunction{≡⟨}\AgdaSpace{}%
\AgdaFunction{cong}\AgdaSpace{}%
\AgdaSymbol{(}\AgdaFunction{boxCalc.rename}\AgdaSpace{}%
\AgdaInductiveConstructor{Sb}\AgdaSymbol{)}\AgdaSpace{}%
\AgdaSymbol{(}\AgdaFunction{pres-single-subst}\AgdaSpace{}%
\AgdaBound{N}\AgdaSpace{}%
\AgdaBound{x}\AgdaSymbol{)}\AgdaSpace{}%
\AgdaFunction{⟩}\<%
\\
\>[2][@{}l@{\AgdaIndent{0}}]%
\>[4]\AgdaFunction{boxCalc.rename}\AgdaSpace{}%
\AgdaInductiveConstructor{Sb}\AgdaSpace{}%
\AgdaSymbol{(}\AgdaFunction{embedTerm}\AgdaSpace{}%
\AgdaSymbol{((}\AgdaFunction{f}\AgdaSpace{}%
\AgdaBound{N}\AgdaSymbol{)}\AgdaSpace{}%
\AgdaBound{x}\AgdaSymbol{))}\<%
\\
%
\>[2]\AgdaFunction{≡⟨}\AgdaSpace{}%
\AgdaFunction{pres-rename}\AgdaSpace{}%
\AgdaInductiveConstructor{S}\AgdaSpace{}%
\AgdaInductiveConstructor{Sb}\AgdaSpace{}%
\AgdaSymbol{(λ}\AgdaSpace{}%
\AgdaBound{x₁}\AgdaSpace{}%
\AgdaSymbol{→}\AgdaSpace{}%
\AgdaInductiveConstructor{refl}\AgdaSymbol{)}\AgdaSpace{}%
\AgdaSymbol{(}\AgdaFunction{f}\AgdaSpace{}%
\AgdaBound{N}\AgdaSpace{}%
\AgdaBound{x}\AgdaSymbol{)}\AgdaSpace{}%
\AgdaFunction{⟩}\<%
\\
\>[2][@{}l@{\AgdaIndent{0}}]%
\>[4]\AgdaFunction{embedTerm}\AgdaSpace{}%
\AgdaSymbol{(}\AgdaFunction{calc.rename}\AgdaSpace{}%
\AgdaInductiveConstructor{S}\AgdaSpace{}%
\AgdaSymbol{(}\AgdaFunction{f}\AgdaSpace{}%
\AgdaBound{N}\AgdaSpace{}%
\AgdaBound{x}\AgdaSymbol{))}\<%
\\
%
\>[2]\AgdaOperator{\AgdaFunction{∎eq}}\<%
\\
%
\\[\AgdaEmptyExtraSkip]%
\>[0]\AgdaComment{--\ Reflection\ and\ pres\ of\ single\ substitution\ of\ refined\ Girard's\ translation\ }\<%
\\
\>[0]\AgdaFunction{pres-subst}\AgdaSpace{}%
\AgdaSymbol{:}\AgdaSpace{}%
\AgdaSymbol{∀}\AgdaSpace{}%
\AgdaSymbol{\{}\AgdaBound{Γ}\AgdaSpace{}%
\AgdaBound{A}\AgdaSpace{}%
\AgdaBound{B}\AgdaSymbol{\}}\AgdaSpace{}%
\AgdaSymbol{(}\AgdaBound{M}\AgdaSpace{}%
\AgdaSymbol{:}\AgdaSpace{}%
\AgdaBound{Γ}\AgdaSpace{}%
\AgdaOperator{\AgdaInductiveConstructor{,}}\AgdaSpace{}%
\AgdaBound{B}\AgdaSpace{}%
\AgdaOperator{\AgdaDatatype{⊢}}\AgdaSpace{}%
\AgdaBound{A}\AgdaSymbol{)}\AgdaSpace{}%
\AgdaSymbol{→}\AgdaSpace{}%
\AgdaSymbol{(}\AgdaBound{N}\AgdaSpace{}%
\AgdaSymbol{:}\AgdaSpace{}%
\AgdaBound{Γ}\AgdaSpace{}%
\AgdaOperator{\AgdaDatatype{⊢}}\AgdaSpace{}%
\AgdaBound{B}\AgdaSymbol{)}\AgdaSpace{}%
\AgdaSymbol{→}\AgdaSpace{}%
\AgdaSymbol{(}\AgdaFunction{embedTerm}\AgdaSpace{}%
\AgdaBound{M}\AgdaSymbol{)}\AgdaSpace{}%
\AgdaOperator{\AgdaFunction{[}}\AgdaSpace{}%
\AgdaFunction{embedTerm}\AgdaSpace{}%
\AgdaBound{N}\AgdaSpace{}%
\AgdaOperator{\AgdaFunction{]b}}\AgdaSpace{}%
\AgdaOperator{\AgdaDatatype{≡}}\AgdaSpace{}%
\AgdaFunction{embedTerm}\AgdaSpace{}%
\AgdaSymbol{(}\AgdaBound{M}\AgdaSpace{}%
\AgdaOperator{\AgdaFunction{[}}\AgdaSpace{}%
\AgdaBound{N}\AgdaSpace{}%
\AgdaOperator{\AgdaFunction{]}}\AgdaSymbol{)}\<%
\\
\>[0]\AgdaFunction{pres-subst}\AgdaSpace{}%
\AgdaSymbol{(}\AgdaInductiveConstructor{`}\AgdaSpace{}%
\AgdaInductiveConstructor{Z}\AgdaSymbol{)}\AgdaSpace{}%
\AgdaBound{N}\AgdaSpace{}%
\AgdaSymbol{=}\AgdaSpace{}%
\AgdaInductiveConstructor{refl}\<%
\\
\>[0]\AgdaFunction{pres-subst}\AgdaSpace{}%
\AgdaSymbol{(}\AgdaInductiveConstructor{`}\AgdaSpace{}%
\AgdaSymbol{(}\AgdaInductiveConstructor{S}\AgdaSpace{}%
\AgdaBound{x}\AgdaSymbol{))}\AgdaSpace{}%
\AgdaBound{N}\AgdaSpace{}%
\AgdaSymbol{=}\AgdaSpace{}%
\AgdaInductiveConstructor{refl}\<%
\\
\>[0]\AgdaFunction{pres-subst}\AgdaSpace{}%
\AgdaSymbol{(}\AgdaInductiveConstructor{ƛ}\AgdaSpace{}%
\AgdaBound{M}\AgdaSymbol{)}\AgdaSpace{}%
\AgdaBound{N}\AgdaSpace{}%
\AgdaSymbol{=}\<%
\\
\>[0][@{}l@{\AgdaIndent{0}}]%
\>[2]\AgdaOperator{\AgdaFunction{beginEq}}\<%
\\
\>[2][@{}l@{\AgdaIndent{0}}]%
\>[4]\AgdaSymbol{(}\AgdaFunction{embedTerm}\AgdaSpace{}%
\AgdaSymbol{(}\AgdaInductiveConstructor{ƛ}\AgdaSpace{}%
\AgdaBound{M}\AgdaSymbol{))}\AgdaSpace{}%
\AgdaOperator{\AgdaFunction{[}}\AgdaSpace{}%
\AgdaFunction{embedTerm}\AgdaSpace{}%
\AgdaBound{N}\AgdaSpace{}%
\AgdaOperator{\AgdaFunction{]b}}\<%
\\
%
\>[2]\AgdaFunction{≡⟨}\AgdaSpace{}%
\AgdaInductiveConstructor{refl}\AgdaSpace{}%
\AgdaFunction{⟩}\<%
\\
\>[2][@{}l@{\AgdaIndent{0}}]%
\>[4]\AgdaInductiveConstructor{ƛb}\AgdaSpace{}%
\AgdaSymbol{(}\AgdaFunction{substb}\AgdaSpace{}%
\AgdaSymbol{(}\AgdaFunction{boxCalc.exts}\AgdaSpace{}%
\AgdaSymbol{(}\AgdaFunction{fb}\AgdaSpace{}%
\AgdaSymbol{(}\AgdaFunction{embedTerm}\AgdaSpace{}%
\AgdaBound{N}\AgdaSymbol{)))}\AgdaSpace{}%
\AgdaSymbol{(}\AgdaFunction{embedTerm}\AgdaSpace{}%
\AgdaBound{M}\AgdaSymbol{))}\<%
\\
%
\>[2]\AgdaFunction{≡⟨}\AgdaSpace{}%
\AgdaFunction{cong}\AgdaSpace{}%
\AgdaInductiveConstructor{ƛb}\AgdaSpace{}%
\AgdaSymbol{(}\AgdaFunction{pres-simultaneous-subst}\AgdaSpace{}%
\AgdaSymbol{(}\AgdaFunction{boxCalc.exts}\AgdaSpace{}%
\AgdaSymbol{(}\AgdaFunction{fb}\AgdaSpace{}%
\AgdaSymbol{(}\AgdaFunction{embedTerm}\AgdaSpace{}%
\AgdaBound{N}\AgdaSymbol{)))}\AgdaSpace{}%
\AgdaSymbol{(}\AgdaFunction{calc.exts}\AgdaSpace{}%
\AgdaSymbol{(}\AgdaFunction{f}\AgdaSpace{}%
\AgdaBound{N}\AgdaSymbol{))}\AgdaSpace{}%
\AgdaSymbol{(}\AgdaFunction{exts-pres-single-subst}\AgdaSpace{}%
\AgdaBound{N}\AgdaSymbol{)}\AgdaSpace{}%
\AgdaBound{M}\AgdaSymbol{)}\AgdaSpace{}%
\AgdaFunction{⟩}\<%
\\
\>[2][@{}l@{\AgdaIndent{0}}]%
\>[4]\AgdaInductiveConstructor{ƛb}\AgdaSpace{}%
\AgdaSymbol{(}\AgdaFunction{embedTerm}\AgdaSpace{}%
\AgdaSymbol{(}\AgdaFunction{subst}\AgdaSpace{}%
\AgdaSymbol{(}\AgdaFunction{calc.exts}\AgdaSpace{}%
\AgdaSymbol{\AgdaUnderscore{})}\AgdaSpace{}%
\AgdaBound{M}\AgdaSymbol{))}\<%
\\
%
\>[2]\AgdaFunction{≡⟨}\AgdaSpace{}%
\AgdaInductiveConstructor{refl}\AgdaSpace{}%
\AgdaFunction{⟩}\<%
\\
\>[2][@{}l@{\AgdaIndent{0}}]%
\>[4]\AgdaFunction{embedTerm}\AgdaSpace{}%
\AgdaSymbol{(}\AgdaInductiveConstructor{ƛ}\AgdaSpace{}%
\AgdaBound{M}\AgdaSpace{}%
\AgdaOperator{\AgdaFunction{[}}\AgdaSpace{}%
\AgdaBound{N}\AgdaSpace{}%
\AgdaOperator{\AgdaFunction{]}}\AgdaSymbol{)}\<%
\\
%
\>[2]\AgdaOperator{\AgdaFunction{∎eq}}\<%
\\
\>[0]\AgdaFunction{pres-subst}\AgdaSpace{}%
\AgdaSymbol{(}\AgdaBound{M}\AgdaSpace{}%
\AgdaOperator{\AgdaInductiveConstructor{·}}\AgdaSpace{}%
\AgdaBound{M₁}\AgdaSymbol{)}\AgdaSpace{}%
\AgdaBound{N}\AgdaSpace{}%
\AgdaSymbol{=}\<%
\\
\>[0][@{}l@{\AgdaIndent{0}}]%
\>[2]\AgdaOperator{\AgdaFunction{beginEq}}\<%
\\
\>[2][@{}l@{\AgdaIndent{0}}]%
\>[4]\AgdaSymbol{(}\AgdaFunction{embedTerm}\AgdaSpace{}%
\AgdaSymbol{(}\AgdaBound{M}\AgdaSpace{}%
\AgdaOperator{\AgdaInductiveConstructor{·}}\AgdaSpace{}%
\AgdaBound{M₁}\AgdaSymbol{))}\AgdaSpace{}%
\AgdaOperator{\AgdaFunction{[}}\AgdaSpace{}%
\AgdaFunction{embedTerm}\AgdaSpace{}%
\AgdaBound{N}\AgdaSpace{}%
\AgdaOperator{\AgdaFunction{]b}}\<%
\\
%
\>[2]\AgdaFunction{≡⟨}\AgdaSpace{}%
\AgdaInductiveConstructor{refl}\AgdaSpace{}%
\AgdaFunction{⟩}\<%
\\
\>[2][@{}l@{\AgdaIndent{0}}]%
\>[4]\AgdaSymbol{(}\AgdaFunction{embedTerm}\AgdaSpace{}%
\AgdaBound{M}\AgdaSymbol{)}\AgdaSpace{}%
\AgdaOperator{\AgdaFunction{[}}\AgdaSpace{}%
\AgdaFunction{embedTerm}\AgdaSpace{}%
\AgdaBound{N}\AgdaSpace{}%
\AgdaOperator{\AgdaFunction{]b}}\AgdaSpace{}%
\AgdaOperator{\AgdaInductiveConstructor{·b}}\AgdaSpace{}%
\AgdaInductiveConstructor{box}\AgdaSpace{}%
\AgdaSymbol{(}\AgdaFunction{embedTerm}\AgdaSpace{}%
\AgdaBound{M₁}\AgdaSymbol{)}\AgdaSpace{}%
\AgdaOperator{\AgdaFunction{[}}\AgdaSpace{}%
\AgdaFunction{embedTerm}\AgdaSpace{}%
\AgdaBound{N}\AgdaSpace{}%
\AgdaOperator{\AgdaFunction{]b}}\<%
\\
%
\>[2]\AgdaFunction{≡⟨}\AgdaSpace{}%
\AgdaFunction{cong}\AgdaSpace{}%
\AgdaSymbol{(}\AgdaOperator{\AgdaInductiveConstructor{\AgdaUnderscore{}·b}}\AgdaSpace{}%
\AgdaInductiveConstructor{box}\AgdaSpace{}%
\AgdaSymbol{(}\AgdaFunction{embedTerm}\AgdaSpace{}%
\AgdaBound{M₁}\AgdaSymbol{)}\AgdaSpace{}%
\AgdaOperator{\AgdaFunction{[}}\AgdaSpace{}%
\AgdaFunction{embedTerm}\AgdaSpace{}%
\AgdaBound{N}\AgdaSpace{}%
\AgdaOperator{\AgdaFunction{]b}}\AgdaSymbol{)}\AgdaSpace{}%
\AgdaSymbol{(}\AgdaFunction{pres-subst}\AgdaSpace{}%
\AgdaBound{M}\AgdaSpace{}%
\AgdaBound{N}\AgdaSymbol{)}\AgdaSpace{}%
\AgdaFunction{⟩}\<%
\\
\>[2][@{}l@{\AgdaIndent{0}}]%
\>[4]\AgdaFunction{embedTerm}\AgdaSpace{}%
\AgdaSymbol{(}\AgdaBound{M}\AgdaSpace{}%
\AgdaOperator{\AgdaFunction{[}}\AgdaSpace{}%
\AgdaBound{N}\AgdaSpace{}%
\AgdaOperator{\AgdaFunction{]}}\AgdaSymbol{)}\AgdaSpace{}%
\AgdaOperator{\AgdaInductiveConstructor{·b}}\AgdaSpace{}%
\AgdaInductiveConstructor{box}\AgdaSpace{}%
\AgdaSymbol{(}\AgdaFunction{embedTerm}\AgdaSpace{}%
\AgdaBound{M₁}\AgdaSymbol{)}\AgdaSpace{}%
\AgdaOperator{\AgdaFunction{[}}\AgdaSpace{}%
\AgdaFunction{embedTerm}\AgdaSpace{}%
\AgdaBound{N}\AgdaSpace{}%
\AgdaOperator{\AgdaFunction{]b}}\<%
\\
%
\>[2]\AgdaFunction{≡⟨}\AgdaSpace{}%
\AgdaFunction{cong}\AgdaSpace{}%
\AgdaSymbol{(λ}\AgdaSpace{}%
\AgdaBound{x}\AgdaSpace{}%
\AgdaSymbol{→}\AgdaSpace{}%
\AgdaFunction{embedTerm}\AgdaSpace{}%
\AgdaSymbol{(}\AgdaBound{M}\AgdaSpace{}%
\AgdaOperator{\AgdaFunction{[}}\AgdaSpace{}%
\AgdaBound{N}\AgdaSpace{}%
\AgdaOperator{\AgdaFunction{]}}\AgdaSymbol{)}\AgdaSpace{}%
\AgdaOperator{\AgdaInductiveConstructor{·b}}\AgdaSpace{}%
\AgdaBound{x}\AgdaSymbol{)}\AgdaSpace{}%
\AgdaSymbol{(}\AgdaFunction{cong}\AgdaSpace{}%
\AgdaInductiveConstructor{box}\AgdaSpace{}%
\AgdaSymbol{(}\AgdaFunction{pres-subst}\AgdaSpace{}%
\AgdaBound{M₁}\AgdaSpace{}%
\AgdaBound{N}\AgdaSymbol{))}\AgdaSpace{}%
\AgdaFunction{⟩}\<%
\\
\>[2][@{}l@{\AgdaIndent{0}}]%
\>[4]\AgdaFunction{embedTerm}\AgdaSpace{}%
\AgdaSymbol{(}\AgdaBound{M}\AgdaSpace{}%
\AgdaOperator{\AgdaFunction{[}}\AgdaSpace{}%
\AgdaBound{N}\AgdaSpace{}%
\AgdaOperator{\AgdaFunction{]}}\AgdaSymbol{)}\AgdaSpace{}%
\AgdaOperator{\AgdaInductiveConstructor{·b}}\AgdaSpace{}%
\AgdaInductiveConstructor{box}\AgdaSpace{}%
\AgdaSymbol{(}\AgdaFunction{embedTerm}\AgdaSpace{}%
\AgdaSymbol{(}\AgdaBound{M₁}\AgdaSpace{}%
\AgdaOperator{\AgdaFunction{[}}\AgdaSpace{}%
\AgdaBound{N}\AgdaSpace{}%
\AgdaOperator{\AgdaFunction{]}}\AgdaSymbol{))}\<%
\\
%
\>[2]\AgdaFunction{≡⟨}\AgdaSpace{}%
\AgdaInductiveConstructor{refl}\AgdaSpace{}%
\AgdaFunction{⟩}\<%
\\
\>[2][@{}l@{\AgdaIndent{0}}]%
\>[4]\AgdaFunction{embedTerm}\AgdaSpace{}%
\AgdaSymbol{((}\AgdaBound{M}\AgdaSpace{}%
\AgdaOperator{\AgdaInductiveConstructor{·}}\AgdaSpace{}%
\AgdaBound{M₁}\AgdaSymbol{)}\AgdaSpace{}%
\AgdaOperator{\AgdaFunction{[}}\AgdaSpace{}%
\AgdaBound{N}\AgdaSpace{}%
\AgdaOperator{\AgdaFunction{]}}\AgdaSymbol{)}\<%
\\
%
\>[2]\AgdaOperator{\AgdaFunction{∎eq}}\<%
\\
%
\\[\AgdaEmptyExtraSkip]%
\>[0]\AgdaComment{--\ Preservation\ of\ reduction\ of\ Girard's\ translation}\<%
\\
\>[0]\AgdaFunction{pres-red}\AgdaSpace{}%
\AgdaSymbol{:}\AgdaSpace{}%
\AgdaSymbol{∀}\AgdaSpace{}%
\AgdaSymbol{\{}\AgdaBound{Γ}\AgdaSpace{}%
\AgdaBound{A}\AgdaSymbol{\}}\<%
\\
\>[0][@{}l@{\AgdaIndent{0}}]%
\>[2]\AgdaSymbol{→}\AgdaSpace{}%
\AgdaSymbol{(}\AgdaBound{M}\AgdaSpace{}%
\AgdaSymbol{:}\AgdaSpace{}%
\AgdaBound{Γ}\AgdaSpace{}%
\AgdaOperator{\AgdaDatatype{⊢}}\AgdaSpace{}%
\AgdaBound{A}\AgdaSymbol{)}\<%
\\
%
\>[2]\AgdaSymbol{→}\AgdaSpace{}%
\AgdaSymbol{(}\AgdaBound{N}\AgdaSpace{}%
\AgdaSymbol{:}\AgdaSpace{}%
\AgdaBound{Γ}\AgdaSpace{}%
\AgdaOperator{\AgdaDatatype{⊢}}\AgdaSpace{}%
\AgdaBound{A}\AgdaSymbol{)}\<%
\\
%
\>[2]\AgdaSymbol{→}\AgdaSpace{}%
\AgdaSymbol{(}\AgdaBound{M}\AgdaSpace{}%
\AgdaOperator{\AgdaDatatype{↝βₙ}}\AgdaSpace{}%
\AgdaBound{N}\AgdaSymbol{)}\<%
\\
%
\>[2]\AgdaSymbol{→}\AgdaSpace{}%
\AgdaSymbol{(}\AgdaFunction{embedTerm}\AgdaSpace{}%
\AgdaBound{M}\AgdaSpace{}%
\AgdaOperator{\AgdaDatatype{↝βb}}\AgdaSpace{}%
\AgdaFunction{embedTerm}\AgdaSpace{}%
\AgdaBound{N}\AgdaSymbol{)}\<%
\\
\>[0]\AgdaFunction{pres-red}\AgdaSpace{}%
\AgdaSymbol{(}\AgdaInductiveConstructor{ƛ}\AgdaSpace{}%
\AgdaBound{M₁}\AgdaSpace{}%
\AgdaOperator{\AgdaInductiveConstructor{·}}\AgdaSpace{}%
\AgdaBound{M₂}\AgdaSymbol{)}\AgdaSpace{}%
\AgdaBound{N}\AgdaSpace{}%
\AgdaInductiveConstructor{βₙ}\AgdaSpace{}%
\AgdaSymbol{=}\<%
\\
\>[0][@{}l@{\AgdaIndent{0}}]%
\>[2]\AgdaFunction{Eq.subst}\AgdaSpace{}%
\AgdaSymbol{(λ}\AgdaSpace{}%
\AgdaBound{t}\AgdaSpace{}%
\AgdaSymbol{→}\AgdaSpace{}%
\AgdaInductiveConstructor{ƛb}\AgdaSpace{}%
\AgdaSymbol{(}\AgdaFunction{embedTerm}\AgdaSpace{}%
\AgdaBound{M₁}\AgdaSymbol{)}\AgdaSpace{}%
\AgdaOperator{\AgdaInductiveConstructor{·b}}\AgdaSpace{}%
\AgdaInductiveConstructor{box}\AgdaSpace{}%
\AgdaSymbol{(}\AgdaFunction{embedTerm}\AgdaSpace{}%
\AgdaBound{M₂}\AgdaSymbol{)}\AgdaSpace{}%
\AgdaOperator{\AgdaDatatype{↝βb}}\AgdaSpace{}%
\AgdaBound{t}\AgdaSymbol{)}\<%
\\
%
\>[2]\AgdaSymbol{(}\AgdaFunction{pres-subst}\AgdaSpace{}%
\AgdaBound{M₁}\AgdaSpace{}%
\AgdaBound{M₂}\AgdaSymbol{)}\<%
\\
%
\>[2]\AgdaInductiveConstructor{βb}\<%
\\
\>[0]\AgdaFunction{pres-red}\AgdaSpace{}%
\AgdaBound{M}\AgdaSpace{}%
\AgdaBound{N}\AgdaSpace{}%
\AgdaSymbol{(}\AgdaInductiveConstructor{μ}\AgdaSpace{}%
\AgdaSymbol{\{}\AgdaArgument{M}\AgdaSpace{}%
\AgdaSymbol{=}\AgdaSpace{}%
\AgdaBound{M₁}\AgdaSymbol{\}}\AgdaSpace{}%
\AgdaSymbol{\{}\AgdaArgument{M'}\AgdaSpace{}%
\AgdaSymbol{=}\AgdaSpace{}%
\AgdaBound{M'}\AgdaSymbol{\}}\AgdaSpace{}%
\AgdaBound{red}\AgdaSymbol{)}\AgdaSpace{}%
\AgdaSymbol{=}\<%
\\
\>[0][@{}l@{\AgdaIndent{0}}]%
\>[2]\AgdaInductiveConstructor{μ}\AgdaSpace{}%
\AgdaSymbol{(}\AgdaFunction{pres-red}\AgdaSpace{}%
\AgdaBound{M₁}\AgdaSpace{}%
\AgdaBound{M'}\AgdaSpace{}%
\AgdaBound{red}\AgdaSymbol{)}\<%
\\
\>[0]\AgdaFunction{pres-red}\AgdaSpace{}%
\AgdaBound{M}\AgdaSpace{}%
\AgdaBound{N}\AgdaSpace{}%
\AgdaSymbol{(}\AgdaInductiveConstructor{ν<}\AgdaSpace{}%
\AgdaSymbol{\{}\AgdaArgument{M}\AgdaSpace{}%
\AgdaSymbol{=}\AgdaSpace{}%
\AgdaBound{M₁}\AgdaSymbol{\}}\AgdaSpace{}%
\AgdaSymbol{\{}\AgdaArgument{N}\AgdaSpace{}%
\AgdaSymbol{=}\AgdaSpace{}%
\AgdaBound{N₁}\AgdaSymbol{\}}\AgdaSpace{}%
\AgdaSymbol{\{}\AgdaArgument{N'}\AgdaSpace{}%
\AgdaSymbol{=}\AgdaSpace{}%
\AgdaBound{N'}\AgdaSymbol{\}}\AgdaSpace{}%
\AgdaBound{x}\AgdaSpace{}%
\AgdaBound{red}\AgdaSymbol{)}\AgdaSpace{}%
\AgdaSymbol{=}\<%
\\
\>[0][@{}l@{\AgdaIndent{0}}]%
\>[2]\AgdaInductiveConstructor{ν}\AgdaSpace{}%
\AgdaSymbol{(}\AgdaInductiveConstructor{ζ}\AgdaSpace{}%
\AgdaSymbol{(}\AgdaFunction{pres-red}\AgdaSpace{}%
\AgdaBound{N₁}\AgdaSpace{}%
\AgdaBound{N'}\AgdaSpace{}%
\AgdaBound{red}\AgdaSymbol{))}\<%
\\
\>[0]\AgdaFunction{pres-red}\AgdaSpace{}%
\AgdaBound{M}\AgdaSpace{}%
\AgdaBound{N}\AgdaSpace{}%
\AgdaSymbol{(}\AgdaInductiveConstructor{ν}\AgdaSpace{}%
\AgdaSymbol{\{}\AgdaArgument{M}\AgdaSpace{}%
\AgdaSymbol{=}\AgdaSpace{}%
\AgdaBound{M₁}\AgdaSymbol{\}}\AgdaSpace{}%
\AgdaSymbol{\{}\AgdaArgument{N}\AgdaSpace{}%
\AgdaSymbol{=}\AgdaSpace{}%
\AgdaBound{N₁}\AgdaSymbol{\}}\AgdaSpace{}%
\AgdaSymbol{\{}\AgdaArgument{N'}\AgdaSpace{}%
\AgdaSymbol{=}\AgdaSpace{}%
\AgdaBound{N'}\AgdaSymbol{\}}\AgdaSpace{}%
\AgdaBound{red}\AgdaSymbol{)}\AgdaSpace{}%
\AgdaSymbol{=}\<%
\\
\>[0][@{}l@{\AgdaIndent{0}}]%
\>[2]\AgdaInductiveConstructor{ν}\AgdaSpace{}%
\AgdaSymbol{(}\AgdaInductiveConstructor{ζ}\AgdaSpace{}%
\AgdaSymbol{(}\AgdaFunction{pres-red}\AgdaSpace{}%
\AgdaBound{N₁}\AgdaSpace{}%
\AgdaBound{N'}\AgdaSpace{}%
\AgdaBound{red}\AgdaSymbol{))}\<%
\\
\>[0]\AgdaFunction{pres-red}\AgdaSpace{}%
\AgdaBound{M}\AgdaSpace{}%
\AgdaBound{N}\AgdaSpace{}%
\AgdaSymbol{(}\AgdaInductiveConstructor{ξ}\AgdaSpace{}%
\AgdaSymbol{\{}\AgdaArgument{M}\AgdaSpace{}%
\AgdaSymbol{=}\AgdaSpace{}%
\AgdaBound{M₁}\AgdaSymbol{\}}\AgdaSpace{}%
\AgdaSymbol{\{}\AgdaArgument{M'}\AgdaSpace{}%
\AgdaSymbol{=}\AgdaSpace{}%
\AgdaBound{M'}\AgdaSymbol{\}}\AgdaSpace{}%
\AgdaBound{red}\AgdaSymbol{)}\AgdaSpace{}%
\AgdaSymbol{=}\<%
\\
\>[0][@{}l@{\AgdaIndent{0}}]%
\>[2]\AgdaInductiveConstructor{ξ}\AgdaSpace{}%
\AgdaSymbol{(}\AgdaFunction{pres-red}\AgdaSpace{}%
\AgdaBound{M₁}\AgdaSpace{}%
\AgdaBound{M'}\AgdaSpace{}%
\AgdaBound{red}\AgdaSymbol{)}\<%
\\
\>[0]\<%
\end{code}
\begin{code}%
\>[0]\AgdaFunction{pres-eval}\AgdaSpace{}%
\AgdaSymbol{:}\AgdaSpace{}%
\AgdaSymbol{∀}\AgdaSpace{}%
\AgdaSymbol{\{}\AgdaBound{Γ}\AgdaSpace{}%
\AgdaBound{A}\AgdaSymbol{\}}\<%
\\
\>[0][@{}l@{\AgdaIndent{0}}]%
\>[2]\AgdaSymbol{→}\AgdaSpace{}%
\AgdaSymbol{(}\AgdaBound{M}\AgdaSpace{}%
\AgdaSymbol{:}\AgdaSpace{}%
\AgdaBound{Γ}\AgdaSpace{}%
\AgdaOperator{\AgdaDatatype{⊢}}\AgdaSpace{}%
\AgdaBound{A}\AgdaSymbol{)}\<%
\\
%
\>[2]\AgdaSymbol{→}\AgdaSpace{}%
\AgdaSymbol{(}\AgdaBound{N}\AgdaSpace{}%
\AgdaSymbol{:}\AgdaSpace{}%
\AgdaBound{Γ}\AgdaSpace{}%
\AgdaOperator{\AgdaDatatype{⊢}}\AgdaSpace{}%
\AgdaBound{A}\AgdaSymbol{)}\<%
\\
%
\>[2]\AgdaSymbol{→}\AgdaSpace{}%
\AgdaSymbol{(}\AgdaBound{M}\AgdaSpace{}%
\AgdaOperator{\AgdaDatatype{↝ₙ}}\AgdaSpace{}%
\AgdaBound{N}\AgdaSymbol{)}\<%
\\
%
\>[2]\AgdaSymbol{→}\AgdaSpace{}%
\AgdaSymbol{(}\AgdaFunction{embedTerm}\AgdaSpace{}%
\AgdaBound{M}\AgdaSpace{}%
\AgdaOperator{\AgdaDatatype{↝bₙ}}\AgdaSpace{}%
\AgdaFunction{embedTerm}\AgdaSpace{}%
\AgdaBound{N}\AgdaSymbol{)}\<%
\\
\>[0]\AgdaFunction{pres-eval}\AgdaSpace{}%
\AgdaSymbol{(}\AgdaInductiveConstructor{ƛ}\AgdaSpace{}%
\AgdaBound{M₁}\AgdaSpace{}%
\AgdaOperator{\AgdaInductiveConstructor{·}}\AgdaSpace{}%
\AgdaBound{M₂}\AgdaSymbol{)}\AgdaSpace{}%
\AgdaBound{N}\AgdaSpace{}%
\AgdaInductiveConstructor{βₙ}\AgdaSpace{}%
\AgdaSymbol{=}\<%
\\
\>[0][@{}l@{\AgdaIndent{0}}]%
\>[2]\AgdaFunction{Eq.subst}\<%
\\
\>[2][@{}l@{\AgdaIndent{0}}]%
\>[4]\AgdaSymbol{(λ}\AgdaSpace{}%
\AgdaBound{t}\AgdaSpace{}%
\AgdaSymbol{→}\AgdaSpace{}%
\AgdaInductiveConstructor{ƛb}\AgdaSpace{}%
\AgdaSymbol{(}\AgdaFunction{embedTerm}\AgdaSpace{}%
\AgdaBound{M₁}\AgdaSymbol{)}\AgdaSpace{}%
\AgdaOperator{\AgdaInductiveConstructor{·b}}\AgdaSpace{}%
\AgdaInductiveConstructor{box}\AgdaSpace{}%
\AgdaSymbol{(}\AgdaFunction{embedTerm}\AgdaSpace{}%
\AgdaBound{M₂}\AgdaSymbol{)}\AgdaSpace{}%
\AgdaOperator{\AgdaDatatype{↝bₙ}}\AgdaSpace{}%
\AgdaBound{t}\AgdaSymbol{)}\<%
\\
%
\>[4]\AgdaSymbol{(}\AgdaFunction{pres-subst}\AgdaSpace{}%
\AgdaBound{M₁}\AgdaSpace{}%
\AgdaBound{M₂}\AgdaSymbol{)}\<%
\\
%
\>[4]\AgdaInductiveConstructor{βb}\<%
\\
\>[0]\AgdaFunction{pres-eval}\AgdaSpace{}%
\AgdaBound{M}\AgdaSpace{}%
\AgdaBound{N}\AgdaSpace{}%
\AgdaSymbol{(}\AgdaInductiveConstructor{μ}\AgdaSpace{}%
\AgdaSymbol{\{}\AgdaArgument{M}\AgdaSpace{}%
\AgdaSymbol{=}\AgdaSpace{}%
\AgdaBound{M₁}\AgdaSymbol{\}}\AgdaSpace{}%
\AgdaSymbol{\{}\AgdaArgument{M'}\AgdaSpace{}%
\AgdaSymbol{=}\AgdaSpace{}%
\AgdaBound{M'}\AgdaSymbol{\}}\AgdaSpace{}%
\AgdaBound{red}\AgdaSymbol{)}\AgdaSpace{}%
\AgdaSymbol{=}\<%
\\
\>[0][@{}l@{\AgdaIndent{0}}]%
\>[2]\AgdaInductiveConstructor{μ}\AgdaSpace{}%
\AgdaSymbol{(}\AgdaFunction{pres-eval}\AgdaSpace{}%
\AgdaBound{M₁}\AgdaSpace{}%
\AgdaBound{M'}\AgdaSpace{}%
\AgdaBound{red}\AgdaSymbol{)}\<%
\end{code}
  %   \end{block}
  % \end{frame}
\end{document}